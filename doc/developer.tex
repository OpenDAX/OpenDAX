\documentclass[10pt,letterpaper]{report}

\usepackage[american]{babel}
\usepackage{amsmath}
\usepackage{amsfonts}
\usepackage{amssymb}
\usepackage{setspace}
\usepackage{parskip}
\usepackage[dvips]{graphicx}
\usepackage{float}
\usepackage{makeidx}
\usepackage{verbatim}


\setlength\parskip{10pt}
\setcounter{secnumdepth}{3}

\makeatletter
\def\thickhrulefill{\leavevmode \leaders \hrule height 1pt\hfill \kern \z@}
\renewcommand{\maketitle}{\begin{titlepage}%
    \let\footnotesize\small
    \let\footnoterule\relax
    \parindent \z@
    \reset@font
    \null\vfil
    \begin{flushleft}
      \huge \@title
    \end{flushleft}
    \par
    \hrule height 1pt
    \par
    \begin{flushright}
      \LARGE \@author \par
    \end{flushright}
    \vskip 60\p@
    \vfil\null
  \end{titlepage}%
  \setcounter{footnote}{0}%
}

\makeatother
\makeindex

\def\opendax{\textit{OpenDAX}}
\def\modbus{\textit{Modbus}$\textsuperscript{\textregistered}$}


\title{OpenDAX Developer's Guide}
\date{July 15, 2008}
\author{Phil Birkelbach}

\begin{document}
\pagenumbering{roman}
\maketitle
\begin{flushleft}
Copyright \textcopyright 2008 - Phil Birkelbach\linebreak
All Rights Reserved

\end{flushleft}

\tableofcontents
\newpage
\pagenumbering{arabic}

\input{dev_intro.tex}
\input{dev_config.tex}


\chapter{Dealing with Data}
The basic unit of data in OpenDAX is the \textit{Tag}.  A tag is similar to a variable in a programming language.
\section{Data Types}
Tags can be one of 15 different data types in OpenDax.  These are given in the following table.

\begin{tabular}{|l|l|l|l|l|}
\hline \textbf{Name} & \textbf{Description} & \textbf{Size (bits)} & \textbf{Min} & \textbf{Max} \\
\hline BOOL & Boolean (True/False) & 1 & 0 & 1 \\
\hline BYTE & Bit String & 8 & 0 & 255 \\
\hline SINT & Signed Short Integer & 8 & -128 & 127 \\
\hline WORD & Bit String & 16 & 0 & 65,535 \\
\hline INT & Signed Integer & 16 & -32,768 & 32,767 \\
\hline UINT & Unsigned Integer & 16 & 0 & 65,535 \\
\hline DWORD & Bit String & 32 & 0 & 4,294,967,296 \\
\hline DINT & Double Integer & 32 & -2,147,483,648 & 2,147,483,647 \\
\hline UDINT & Unsigned Double Integer & 32 & 0 & 4,294,967,296 \\
\hline TIME & Unix Timestamp & 32 &  &  \\
\hline REAL & IEC 754 Floating Point & 32 &  &  \\
\hline LWORD & Bit String & 64 & 0 & $2^{64}$ \\
\hline LINT & Long Integer & 64 & $-2^{63}$ & $2^{63}-1$ \\
\hline ULINT & Unsigned Long Integer & 64 & 0 & $2^{64}$ \\
\hline LREAL & IEC 754 Double Floating Point & 64 &  &  \\
\hline 
\end{tabular}
 
The opendax.h header file contains the definitions for these data types.  These definitions are the name of the data type in the above table prefixed with "\texttt{DAX\_}".  So to represent a Boolean data type to the OpenDAX library you would use the definition \texttt{DAX\_BOOL}.  These definitions are used anytime your module needs to communicate a data type to the OpenDAX library.  For example, when creating a tag the module has to specify the data type.  You could create an INT tag with a call like this...

\begin{verbatim}
dax_add_tag("MyInt", DAX_INT, 1);
\end{verbatim}

This would create a tag in the server with the name \textit{MyInt} of type INT.  The 1 as the last argument just means a single member.  A larger number would signify an array.

There are also some typedefs in opendax.h that help with declaring variables within you module code.  They are the same as the precompiler defines except they are all lower case.  These make sure that the variable definitions inside your module match the data types of the tags in OpenDAX.  For the above Tag you could create a variable in C with this code...

\begin{verbatim}
dax_int myInt;
myInt = 13000;
\end{verbatim}

Each Tag in OpenDAX can be a single value of any of these base data types or it can be an array of these.  

\begin{verbatim}
dax_add_tag("MyInt", DAX_INT, 10);
\end{verbatim}

This code would generate an array of INT's in the Server.  It is important to note that tags can be redefined as long as they data type stays the same.  If you call dax\_add\_tag() again with a count that is higher than the previous call it would increase the size of the array.  If you call it with a count smaller than the previous, it will ignore it and keep the array the same size.  If you change the data type in the second call the function will return an error.

OpenDAX also supports the concept of a \textit{Compound Data Type}\index{Compound Data Type}.  This is an aggregate data type that is very similar to a structure in C.  Your module can create a new compound data type or it can use ones that are created by other modules.

These are the four functions that we will need to use from the libdax library.

\begin{verbatim}
dax_cdt *dax_cdt_new(char *name, int *error);

int dax_cdt_member(dax_state *ds, dax_cdt *cdt, char *name,
                   tag_type mem_type, unsigned int count);

int dax_cdt_create(dax_state *ds, dax_cdt *cdt, tag_type *type);

void dax_cdt_free(dax_cdt *cdt);
\end{verbatim}

To create a CDT you first have to allocate a CDT object.

The \verb|dax_cdt_new()|\index{dax\_cdt\_new() function} allocates, initializes and returns a pointer to a new CDT object.  The \textit{name} argument is the name that would be given to the CDT.  The \textit{error} argument is a pointer to an integer that can indicate any errors.  If there is an error the function will return NULL and the integer pointed to \textit{error} will be set to the error code.  If you are not interested in this error code then you can pass NULL to the function.

Once you have the object you add members to the CDT one at a time.  These members can be of any previously defined data type including other CDTs.  They can also be arrays and even arrays of other CDT's.  We do this with the \verb|dax_cdt_member()|\index{dax\_cdt\_member() function}.  The \textit{cdt} argument is the object that was returned from the \verb|dax_cdt_new()| function.  The \textit{name} argument is the name that we want to give to our member.  The \textit{mem\_type} argument is either one of the 15 base DAX\_* datatypes from above or a predefined compound data type.

Once all of the members have been added the module the \verb|dax_cdt_create()|\index{dax\_cdt\_create() function} function is used to send the data type definition to the server and actually create the data type.  The \textit{type} argument to \verb|dax_cdt_create()| will contain the new type identifier of the created compound data type.  This identifier can be used anywhere another data type definition (such as DAX\_DINT) could be used.

The \verb|dax_cdt_free()|\index{dax\_cdt\_free() function} function simply frees the memory associated with the new data type.  Don't try to free the memory yourself because there are other data structures in the \verb|dax_cdt| object that have to be freed.  Simply passing the pointer to \verb|free()| for example will result in a memory leak.  Once the data type has been created it can be freed.  Obviously failure to do this will also result in a memory leak.  Don't try to reuse a \verb|dax_cdt| object either.  You can reuse the pointer but you need to free the old one with \verb|dax_cdt_free()| and then reallocate a new one with \verb|dax_cdt_new()|.

Perhaps it's time for an example.  Let's build a CDT that has the following structure...

\begin{verbatim}
MyCDT
  MyInteger INT
  MyReals   REAL[10]
  MyBools   BOOL[8]
\end{verbatim}

The following code would be used\ldots
\begin{verbatim}
dax_cdt *dc;
int error;
tag_type type;

dc = dax_new("MyCDT", &error);
if(dc == NULL) {
    printf("Error code returned is %d\n", error);
    //handle error
} else {
    dax_cdt_member(ds, dc, "MyInteger", DAX_INT, 1);
    dax_cdt_member(ds, dc, "MyReals", DAX_REAL, 10);
    dax_cdt_member(ds, dc, "MyBools", DAX_BOOL, 8);

    error = dax_cdt_create(ds, dc, &type);
    if(error) {
        printf("Error returned from dax_cdt_member() is %d\n", error);
	    //handle error
    } else {
        printf("CDT created!  Data Type = 0x%X\n", type);
    }
    dax_cdt_free(dc);
}

\end{verbatim}

This code will build the above CDT.  There is some error checking missing on the return values of \verb|dax_cdt_member()| functions but this is for clarity.  In your code you would want to check these return values.  Also be careful of the error handling so that you don't exit this code before the \verb|dax_cdt_free()| function has been called to free the \verb|dax_cdt| object.

\section{Creating Tags}

\section{Reading Data}

\section{Writing Data}


\chapter{Handling Events}

Event handling is not part of the system yet.  This is still a work in progress.

\chapter{Messaging to Other Modules}

Module messaging is not part of the system yet.  This is still a work in progress.

\chapter{Shell Modules}

Shell Modules are called that because they are processes that are normally started from the command line shell.  These modules could be anything from an mp3 player to a database client.  They could be just about any program that can be started from the shell prompt.  They obviously don't have any "normal" OpenDAX functionality.  OpenDAX allows their use by letting other modules gain access to the shell modules STDIN, STDOUT and STDERR file descriptors.  Text can be sent from other modules to these shell modules and controlled as though that text was being typed on the command line.  This allows OpenDAX to easily add functionality found in other programs and perhaps not reinvent too many wheels.

Shell modules must be started by the \opendax server.  If you think about it this only makes sense.  If you simply ran them from the command line they would just be the same old programs.  When they are run as children of the \opendax server, the server would have access to the file descriptors that it needs to manipulate these programs.  It becomes a wrapper process for these programs.

These would be configured in \textit{opendax.conf}.

\chapter{Lua Modules}

It is possible to write an OpenDAX module entirely in Lua\footnote{This feature is only partially implemented at this point but rudimentary support is included and development is ongoing}.  Included in the OpenDAX distribution is a library for interfacing to the OpenDAX API from Lua.  Since Lua is such an integral part of so much of the OpenDAX system this library was written for convenience, but since Lua is such an elegant language to embed and interface with it was easy to take this library and make a Lua Module\footnote{Not to be confused with an OpenDAX module} out of it. 

\section{Installation}

If you installed OpenDAX according to the instructions in this manual you should have the libdaxlua library file installed where the other libraries are installed.  Normally this would be /usr/local/lib but if you changed your prefix to the autoconf script it would be there instead.

To get the module to work the library needs to exist in the path that LUA will look for libraries to load as modules.  In Lua this is passed as the LUA\_CPATH environment variable.  You can either pass this environment variable to all of your scripts or you can just create a symbolic link in one of the directories that are part of the default 'cpath.'  To find out what the cpath is use this command at the command prompt...

\verb|lua -e "print(package.cpath)"|

On my system it looks like this...

\verb|./?.so;/usr/local/lib/lua/5.1/?.so;/usr/local/lib/lua/5.1/loadall.so|

Each path is separated by a ';' and the ? will be replaced with whatever you use as the argument to \textit{require()}.  So to load the luadax module as 'dax' then the Lua interpreter needs to find the library as one of those filenames.  We want the library to load when we use the command \verb|require("dax")| in our Lua scripts.  Add a symbolic link in \textit{/usr/local/lib/lua/5.1/} named \textit{dax.so} that points to the library, \textit{/usr/local/lib/libdaxlua.so}\footnote{In OSX the library will ed in .dylib but the symbolic link will still need to match the pattern in cpath}.

Now when we use \verb|require("dax")| in our Lua scripts it looks in the directory \textit{/usr/local/lib/lua/5.1/} and finds a link to the library.  It loads the library and off we go.

At some point we'll spend some time with the autoconf scripts and try to automate this process but for now this will have to do.

\section{API Description}

All of the functions of the OpenDAX Lua Module that you use to write your OpenDAX module are contained in a table of the same name that you used to create the module.  In my case it's \textit{"dax"}.

\begin{verbatim}
dax.init
\end{verbatim}
\index{dax.init function}

\begin{verbatim}
dax.free
\end{verbatim}
\index{dax.free function}

\begin{verbatim}
dax.cdt_create
\end{verbatim}
\index{dax.cdt_create function}

\begin{verbatim}
dax.tag_add
\end{verbatim}
\index{dax.tag_add function}

\begin{verbatim}
dax.tag_get
\end{verbatim}
\index{dax.tag_get function}

\begin{verbatim}
dax.tag_read
\end{verbatim}
\index{dax.tag_write function}


\appendix

\input{dev_api.tex}

\chapter{Skeleton Module}
The Skeleton module is included in the distrubtion as a starting place for building new modules.  It includes just enough code to function as an OpenDAX module.  The code is well documented and when used with this manual it will, hopefully, make writing a new module for OpenDAX a fairly simple and straight forward process.  The Skeleton module is located in the /modules/skel subdirectory of the distribution.

\section{skel.h}
\begin{small}
\verbatiminput{../modules/skel/skel.h}
\end{small}
\section{skel.c}
\begin{small}
\verbatiminput{../modules/skel/skel.c}
\end{small}

\chapter{Codebase Descriptions}
This section describes the different files and directories that are part of the OpenDAX core programs.  This includes the master daemon, the server and the library.  This is meant as a rough overview to get a feel for how the code is laid out.  It is not meant to be an exhaustive reference.  For that you'll just have to go take a look at the latest code in the repository.  As a module developer you may not need any of this information, but since OpenDAX is such a young program there will still be plenty of bugs to work out of the core system, and this list may help somebody to get into the core code and help out with debugging or development.

All of the source code files are in the /src directory.  Assume that all of the following files and directories are within the /src directory.

\emph{config.h}
This file is generated by ./configure and contains the architecture dependent defines that tell us things like what functions are available and what header files we need to install. I'm actually checking for quite a few functions that I don't really do anything about if they are missing. As incompatibilities come up these will have to be dealt with.

\emph{common.h}

This header includes config.h as well as some other headers that are popular. It also has some definitions and macros that will be used throughout the system. This file should be included in just about every source code file in the system.

\emph{opendax.h}

This is the header that describes the public interface to the OpenDAX library. It contains declarations for all of the public library functions as well as the precompiler definitions for the data types, error codes, configuration flags etc. This file should be included by all modules that will link to the library, and is included in most of the source code files in the rest of the system too. This should be the only header file from this distribution that would need to be included in any module code. If there are others then we did something wrong in the interface.

\emph{libcommon.h}

This header file contains definitions and macros that are needed by both the server and the module library.

\emph{/master}

The master directory contains the source for the master daemon.  This program is responsible for managing all of the other programs that are part of the OpenDAX system.

\emph{/master/master.c}

Contains the main() function and starts the system.

\emph{/master/daemon.c}

This is the code that handles sending the program to the background.

\emph{/master/mstr\_config.c}

The main configuration code is in this file.  It uses a very similar setup as the module library in that, it uses Lua to read configuration files.

\emph{/master/process.c}

Contains the process starting and stopping code as well as any other code that is responsible for dealing with the processes that are running.

\emph{/master/logger.c}

logger.c contains the code to deal with logging messages to the console, syslog or whatever other logging mechanism we decide to use.

\emph{/master/cpu.c}

Code for getting the cpu usage and status from running processes.

\emph{/master/pipes.c}

Pipes are set up between the running processes and the master daemon.  These pipes are the mechanism that modules use to log messages as well as a way for the master daemon to monitor when the modules have started.  There is a separate thread that runs to deal with this and the code for that thread as well as any support functions are in this file.

\emph{/server}

The dax directory contains the source for the OpenDAX server.

\emph{/server/daxtypes.h}

This contains the private type definitions that are used internally by the server. These definitions should not be used by any module or the library.

\emph{/server/server.c}

Contains main() and the other functions necessary to start the server and spawn the threads that do all the work. It all starts here.

\emph{/server/func.c}

Contains a few generic functions for common operations like memory allocation and such.

\emph{/server/module.c}

Contains the code for the module handling system in the server. Operations such as starting and stopping modules as well as module registration and any other operation that involves modules should be in this file. The module.h header contains the public interface definitions for the functions in this file.

\emph{/server/message.c}

This file contains the functions that handle the module<->server messaging. Very little actual work gets done here other than sending and receiving data on the sockets and determining which functions in other files should be called.

\emph{/server/buffer.c}

Contains buffering code for the messaging subsystem. The messaging system writes the data from each socket into these buffers until it figures out that it has an entire message from one of the modules then it calls the function to deal with that message.

\emph{/server/options.c}

Contains the code for reading the configuration from the configuration file and the command line.

\emph{/server/tagbase.c}

The storage and manipulation of all the real time tag information is contained within this file. For now the custom datatype handling code is also in here but this may get moved to it's own file.

\emph{/lib}

The lib directory contains the source for the library. We use libtool to sort out the compatibility issues associated with the way different systems handle shared libraries. If no shared libary system will work then libtool makes this a static library.

\emph{/lib/libdax.h}

This header file contains all of the private definitions, macros and declarations that are needed throughout the library code but are not needed in the server or the modules.

\emph{/lib/libfunc.c}

Contains some generic functions that are useful throughout.

\emph{/lib/libmsg.c}

This is all the messaging code for the library. There are functions in this file for sending and retrieving messages and determining what to do with them. The functions here closely resemble functions in /dax/message.c and typically changes in one of thse files requires changes in the other. Between the two of them they define the
communications protocol.

\emph{/lib/data.c}

This file deals with library side of the tag data. It is probably not named well and may change. The functions here resemble functions that are in /dax/tagname.c

\emph{/lib/libconv.c}

This file contains the functions for making sure that the data formatting is the same as the server. The way that OpenDAX handles different byte ordering and data formating between architectures over the network is that the server stores the data in whatever way the server wants and the library is responsible for determining if the data needs to be converted and how. This file contains the code for that.

\emph{/lib/libcdt.c}

OpenDAX allows the creation of custom data types. This file contains the code to handle all of that.

\emph{/lib/libopt.c}

Contains the functions for configuring the module. The entire configuration system of OpenDAX uses Lua as the configuration programming language. The modules can be configured by either the main opendax.conf file, their own configuration file, the command line or any combination of the above. There are also some configuration options that are common between modules. This file contains the code to handle all of that.

\emph{/modules}

This directory contains the module code. We will not discuss the module details here.

\emph{/etc}

The sample configuration files are located here.

\chapter{opendax.h Listing}
\begin{small}
\verbatiminput{../opendax.h}
\end{small}

%\chapter{GPL License}
%\verbatiminput{../COPYING}

\printindex
\end{document}
