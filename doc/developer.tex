\documentclass[10pt,letterpaper]{report}

\usepackage[american]{babel}
\usepackage{amsmath}
\usepackage{amsfonts}
\usepackage{amssymb}
\usepackage{setspace}
\usepackage{parskip}
\usepackage[dvips]{graphicx}
\usepackage{float}
\usepackage{makeidx}
\usepackage{verbatim}
\usepackage{color}
\usepackage{hyperref}
\hypersetup{
    colorlinks=true, % make the links colored
    linkcolor=blue, % color TOC links in blue
    urlcolor=red, % color URLs in red
    linktoc=all % 'all' will create links for everything in the TOC
}

\setlength\parskip{10pt}
\setcounter{secnumdepth}{3}

\makeatletter
\def\thickhrulefill{\leavevmode \leaders \hrule height 1pt\hfill \kern \z@}
\renewcommand{\maketitle}{\begin{titlepage}%
    \let\footnotesize\small
    \let\footnoterule\relax
    \parindent \z@
    \reset@font
    \null\vfil
    \begin{flushleft}
      \huge \@title
    \end{flushleft}
    \par
    \hrule height 1pt
    \par
    \begin{flushright}
      \LARGE \@author \par
    \end{flushright}
    \vskip 60\p@
    \vfil\null
  \end{titlepage}%
  \setcounter{footnote}{0}%
}

\makeatother
\makeindex

\def\opendax{\textit{OpenDAX}}
\def\modbus{\textit{Modbus}$\textsuperscript{\textregistered}$}
\def\daxstate{\texttt{dax\_state} }
\def\eventadd{\texttt{dax\_event\_add()}}
\def\eventdel{\texttt{dax\_event\_del()}}
\def\eventwait{\texttt{dax\_event\_wait()}}
\def\eventpoll{\texttt{dax\_event\_poll()}}
\def\eventgetfd{\texttt{dax\_event\_get\_fd()}}
\def\eventdispatch{\texttt{dax\_event\_dispatch()}}


\title{OpenDAX Module Developer's Guide}
\date{July 15, 2008}
\author{Phil Birkelbach}

\begin{document}
\pagenumbering{roman}
\maketitle
\begin{flushleft}
Copyright \textcopyright 2008 - Phil Birkelbach\linebreak
All Rights Reserved

\end{flushleft}

\tableofcontents
\newpage
\pagenumbering{arabic}

\chapter{Introduction}
\opendax is an open source, modular, data acquisition and control system. It is licensed under the GPL (GNU General Public License) and therefore is completely free to use and modify.

This book is a tutorial for developing modules for OpenDAX as well as a reference for experienced module developers.  It should be noted that, as of this writing, OpenDAX is very immature and much of the information in this book may have changed.  The developers attempt to keep the interface constant, but at this stage of development it will make sense from time to time to make changes that would break existing module code as well as conflict with information in this book.  The current source code is the ultimate authority on the API.


\section{OpenDAX Basics}
Before we get started, it would be good to discuss how OpenDAX works.  OpenDAX is made up of three main parts, the tag server, the library and the modules.

The OpenDAX tag server\index{OpenDAX server} is the heart and soul of the OpenDAX system.  It sits at the center of the OpenDAX universe and coordinates all of the data and communications of the system.  The libdax library\index{libdax library} abstracts the communication interface to the modules and the modules are where all the work is done.

If the server is the heart and soul of the OpenDAX application then modules are the arms and legs.  Modules do all of the work.  They are separate processes.  Modern operating systems do a great job of managing processes and the OpenDAX developers did not see any need to reinvent that wheel.  

Client Modules handle all of interface to the outside world.  The application logic, any storage or logging functions and the human interface are all handled by modules.  The modules all communicate to the OpenDAX server through an API that is exposed in the libdax library.  The libdax library API is what the OpenDAX module developer will see of OpenDAX.

The low level communications to the OpenDAX server takes place through a BSD Socket interface.  There is no requirement that the module be running on the same machine as the server.  The entire application can be distributed however the application developer desires.  If the module is on the same machine, it can communicate to the server via UNIX Domain Socket.  If the module is on a remote host then it will have to use a TCP Socket.  The UNIX Domain sockets are considerably faster than TCP since they are really nothing more than a memory copy within the kernel.

The exact nature of the communication is subject to change at this point so I won't go into that in too much detail here.  Right now there are two sockets that are created for each module.  The primary socket allows the module to send and receive data to the server, the second socket is used for asynchronous event reporting.

At some point in the future we hope to implement some server-to-server communication that might facilitate redundancy or an even better ability to distribute the system.  Right now there is only one server in any application and all the modules communicate to that single server.

The server contains the tag database.  A tag\index{tag} is the atomic unit of data in the system.  These tags are analogous to variables in a programming language.  There are many different data types in OpenDAX and the user or module programmer can create compound data types that are collections of other types.  Compound data types are similar to a structure in C.  We will discuss compound data types later in this text.

The tag database contains the names of these tags, their data type\index{data type}, the actual real time value and the events that the tag responds too.  It is the central store house of information in OpenDAX.  Different modules work with tags in different ways.  For instance, the \modbus \index{Modbus}module reads data from one or more \modbus devices and stores that information in tags within the server.  The tags are arrays of type UINT or BOOL\footnote{UINT is an unsigned 16 bit data type, BOOL is a single bit data type} depending on the command.  The names of these tags are a configurable parameter of the \modbus module.  If and HMI or Logic module need the information from the \modbus module they would read these tags to get it.

The server also contains information about the currently connected modules.  Modules must register with the server when they are started before the server will answer any queries by the module.  This registration step is how the module and the server learn what they need to learn about each other to properly communicate.  Once registered the server can keep track of that module through the file descriptor of the socket on which the the connection is made.  This information can be made available to other modules that may need it.

There is another program that is included in the OpenDAX distribution that is optional for the system to function but it's very useful.  It is the master program.  (This is the program that we actuall call 'opendax')  Modules can be started by the OpenDAX master or by any other means that the operating system has for starting processes (i.e. shell prompt, scripts etc).  If the module is to be started by the master there are a few advantages.  First the master will know if the module dies for some reason because the operating system will send it a signal.  This allows the master to restart the module or alert the user that a part of the system is down.

\section{Installation}

OpenDAX uses the CMake build system generator.  You'll need to install CMake
on your system.

You will also need the Lua development libraries installed.  Most 
distributions have versions of Lua that will work.  The currently supported
versions of Lua are 5.3 and greater.
If you install Lua from the source files you will need to add -FPIC 
compiler flag to the build.

\begin{verbatim}
make MYCFLAGS="-fPIC" linux
\end{verbatim}

Once you have CMake and the Lua libraries installed you can download and build
OpenDAX.  First clone the repository...

\begin{verbatim}
git clone https://github.com/OpenDAX/OpenDAX.git
\end{verbatim}

This should create the OpenDAX directory.  Now do the following...
\begin{verbatim}
mkdir build
cd build
cmake ..
make
make test
\end{verbatim}

If all the tests pass you can install with \ldots

\begin{verbatim}
sudo make install
sudo ldconfig
\end{verbatim}


\section{Setting up the Build Environment}
The build environment for developing modules is pretty simple.  If you have installed OpenDAX then you should have everything that you need to compile and run OpenDAX modules.

There is no separate development package for \opendax.  All the files that you need to develop modules should be installed when you install \opendax from the distribution.

The library should be installed in a typical place (usually /usr/local/lib) and the \textit{opendax.h}\index{opendax.h} header file should be in a place where your compiler can find it (usually /usr/local/include).  For writing a module in C this should be all that you need.  If you have problems with the installation, see the \emph{OpenDAX User's Manual}, it has much more detailed information on what is needed to get OpenDAX up and running on your system.

You should be able to use any C compiler to build modules for OpenDAX, but we have been using GCC for the main development.

For all intents and purposes your modules sole interaction with the \opendax system is through the libdax library.  You should include the \emph{opendax.h} header file in your module's source code file and you should link the libdax library with your module with the \verb|-ldax| option to the compiler.\footnote{This is for \texttt{gcc}. Other compilers may have different options for linking shared libraries.}

Lua and Python bingings also exist for OpenDAX.  The Lua library is distributed with OpenDAX and the Python bindings called \textit{PyDax} are in a separate repository.

\chapter{Initialization and Configuration}


\section{Initialization}

The first thing that any module will have to do is allocate and initialize a
\verb|dax_state|\index{dax\_state object} object.  This is simply a matter of
calling the \verb|dax_init()|\index{dax\_init() function} function.  For example
see the following code.

\begin{verbatim}
dax_state *ds;

ds = dax_init("MyModule");
if(ds == NULL) exit(-1);
\end{verbatim}

First we declare a pointer to a \verb|dax_state| object.  This is an opaque
pointer that your program will need to maintain for it's entire lifetime.  It
represents a single connection to a single server.\footnote{Modules may connect to multiple servers
	but they will need to maintain multiple dax\_state objects and configurations.}

Once the pointer is declared we use the \verb|dax_init()| function to allocate
it and initialize it.  The only real error that this function can have is
memory allocation failure, in which case it will return NULL.  Your module
should probably log an error an exit gracefully.  The string that is passed to
\verb|dax_init()| is the name of your module as it will be seen throughout the
system.  It can be overridden by names gathered during the configuration
process.  We'll talk about this later.


\section{Configuration}

The second thing that the module code should do is read it's configuration.  It
is important to note that the module will have to read some configuration even
if the module itself does not have any configurable options. This is because the
libdax library will need some configuration information to help it know how to
communicate with the server.

There are two methods for the module to receive configuration information.
A configuration file may or may not exist for each program or arguments can
be passed via the command line.

If a configuration
attribute is present on the command line it will take precedence over that same
attribute given in the configuration files.  An exception to this is the logging
configuration.  Once logging services are added via the configuration file those
services will determine how messages are logged and anything passed on the 
command line will be ignored.

Configuration files are Lua scripts which makes the configuraiton system in
OpenDAX very powerful.  It allows configuration files to import and excecute 
other Lua configuration files.  This gives the user a way to set up universal 
configuration options that are common accross the entire system.  It also allows
the configuration to include variables, loops and conditions that add a lot
of power to configuration.

Basic configuration is made up of attributes.  An example of an attribute would be the
name of the local unix domain socket that the module should use to communicate
to the server.  The attribute name might be \texttt{socketname} and the value
\texttt{/tmp/opendax}\footnote{In fact these are the defaults for the local
	socket configuration}.  There are many attributes that are built into the system
that will always need to be configured.  These are necessary for the basic
functionality.  The \texttt{socketname} attribute is just one example.  The
module developer can add as many attributes as are needed for configuring the
functionality of the given module.  The libdax library contains interfaces to
help the module developer with all of this.

If the configuration data is more complex than simply assigning a value to an
attribute there is a callback function mechanism built into the configuration
system.  Since the configuration files in OpenDAX are nothing more than Lua scripts.
The libdax library gives the module programmer the ability to create functions
that can be called from the script.  These functions are programmed in the
module code just like any other Lua extension function.  The full description of
how to write Lua extension functions is beyond the scope of this book, but we
will give some simple examples that would help with basic configuration.


\subsection{Creating Attributes}
Attributes\index{Attributes} are the names that
are given the different options that you can create in an OpenDAX configuration.
For instance, your \textit{fooserial} module will need to know what serial port
to use, what baud rate to use and the communications settings it will need.
These would be attributes that you would like for the libdax library
configuration system to find for you.

To add and attribute to the configuration state we use the
\verb|dax_add_attribute()|\index{dax\_add\_attribute() function} function.  Here
is the prototype.

\begin{verbatim}
int dax_add_attribute(dax_state *ds, char *name, char *longopt,
                      char shortopt, int flags, char *defvalue);
\end{verbatim}

The \textit{name} argument is the name that will be used for the attribute.  It
would be the left side of the attribute assignment in the configuration file.
For instance if we use \texttt{"baudrate"} for our attribute name the
configuration file might look like\ldots

\begin{verbatim}
baudrate = 9600
\end{verbatim}

The \textit{longopt} argument is the long name of command line argument that
would represent this attribute.  If we use \texttt{"baud-rate"} for
\textit{longopt} then we might set the baud rate by the following command\ldots

\begin{verbatim} $ fooserial --baud-rate=9600 \end{verbatim} %$ The
\textit{shortopt} argument is a single character that works just like
\textit{longopt}\footnote{For more information on how these two types of command
line options work, refer to the documentation for the \textit{getopt} library.}

The \textit{defvalue} argument is the default value that will be assigned to
this attribute if it is not set by any of the three configuration mechanisms.

The \textit{flags} argument is a bitwise OR {|} of the following definitions
\ldots

\begin{tabular}{|l|l|}
\hline \verb|CFG_ARG_NONE| & No Arguments \\
\hline \verb|CFG_ARG_REQUIRED| & Argument is required \\
\hline \verb|CFG_ARG_OPTIONAL| & Argument is optional \\
\hline \verb|CFG_CMDLINE| & Parse Command line \\
\hline \verb|CFG_MODCONF| & Parse [module].conf file \\
\hline \verb|CFG_NO_VALUE| & Don't store a value, only call callback \\
\hline
\end{tabular}

The first three flags correspond to the \textit{getopt} library's usage of
command line arguments.  If there should be no arguments to the attribute then
\verb|CFG_ARG_NONE| should be used.  This would come in handy if the argument is
simply a flag of some kind.  The -V option to many programs simply cause the
program to print it's Version number and exit, is one example of an option that
would take no arguments.

If an argument is required the flag \verb|CFG_ARG_REQUIRED| should be used.  The
baud rate option in the above example would be pretty meaningless without a
number of some kind to use as the baud rate.

The \verb|CFG_ARG_OPTIONAL| flag is used for an attribute that doesn't need an
argument but where one might make sense.  A verbosity option when used alone
might simply increase the verbosity of the programs output slightly but passing
a numerical argument would increase the verbosity by that amount.

The next three flags, \verb|CFG_CMDLINE| and
\verb|CFG_MODCONF| have to do with which of the two configuration sources this
particular attribute could be found.  It may make sense to search for an
attribute on the command line, and/or in our
\textit{fooserial.conf} configuration file.  In this case we would put both
values in our \textit{flags} argument.  It may however only make sense to see
command line options.  In this case we would simply use \verb|CFG_CMDLINE|.

The final flag,\verb|CFG_NO_VALUE| can also be OR'd with the others.  This is
used in the special case where we are not actually interested in any value that
might be passed to the attribute in a configuration file.  The configuration
system has the ability to call a callback function when attributes are set in
configuration files.  If this callback function is the only thing that we are
interested in, we can use this flag to save a little memory.

There are some attributes that are predefined for use by the libdax library.
You cannot use any of the names, long options or short options of these
attributes.  These are listed in the following table.

\begin{tabular}{|l|l|c|}
\hline \textbf{name} & \textbf{longopt} & \textbf{shortopt} \\
\hline socketname & socketname & \texttt{U} \\
\hline serverip & serverip & \texttt{I} \\
\hline serverport & serverport & \texttt{P} \\
\hline server & server & \texttt{S} \\
\hline name & name & \texttt{N} \\
\hline cachesize & cachesize & \texttt{Z} \\
\hline msgtimeout & msgtimeout & \texttt{O} \\
\hline logtopics\footnotemark & logtopics & \texttt{T} \\
\hline verbose\footnotemark[\value{footnote}] & verbose & \texttt{v} \\
\hline config\footnotemark[\value{footnote}] & config & \texttt{C} \\
\hline confdir\footnotemark[\value{footnote}] & confdir & \texttt{K} \\
\hline
\end{tabular}
\footnotetext{These are command line only attributes}

If your module tries to use any of these names or options the
\verb|dax_add_attribute()| function will return an error.  This list is also
subject to change.  If you want to know the absolute latest version of this list
see the \textit{/lib/libopt.c} source code file in the \opendax distribution.

\subsection{Creating Callbacks}
\begin{verbatim}
int dax_attr_callback(dax_state *ds, char *name,
                      int (*attr_callback)(char *name, char *value));
\end{verbatim}

The \verb|dax_attr_callback()|\index{dax\_attr\_callback() function} function is
used to add a callback function to the configuration system that will be called
when this attribute is set.

[[Still working on this]]

\subsection{Writing Lua Functions}

\begin{verbatim}
int dax_set_luafunction(dax_state *ds, int (*f)(void *L), char *name);
\end{verbatim}

The \verb|dax_set_luafunction()|\index{dax\_set\_luafunction() function}
function is used to set a function that can be called from your module
configuration script.  This gives your module a lot of power in how it is
configured.  A full explanation of writing Lua functions is beyond the scope of
this book.  Review the Lua documentation for more information.

[[Still working on this]]

\subsection{Running the Configuration}
To execute the configuration use the following function \ldots
\begin{verbatim}
int dax_configure(dax_state *ds, int argc, char **argv, int flags);
\end{verbatim}

The \verb|dax_configure()|\index{dax\_configure() function} function will run
the configuration.  You pass this function the \textit{argc} and \textit{argv}
variables that were passed to your module in \verb|main()|.

The \textit{flags} argument is a bitwise OR of \verb|CFG_CMDLINE| or 
\verb|CFG_MODCONF|.  These do just what you would think
they would do.  Depending on which of these flags that you set the corresponding
configuration mechanism will be used.  To cause the module to only read from the
command line you would simply set the \verb|CFG_CMDLINE| flag. If you want
the configuration file to be run you would set \verb|CFG_MODCONF| as well.

It's unusual to not use the command line but it's not uncommon to use only
the command line.\footnote{The \verb|daxc| command line client module does not read
a configuration file.}  Most modules will want the full power of the Lua
configuration engine and since the name and or location of that file can change
with options on the command line it makes sense to do both in that situation.

\subsection{Retrieving Attributes}

Once we have run the configuration we use the
\verb|dax_get_attr()|\index{dax\_get\_attr() function} function to retrieve the
values that were set.

\begin{verbatim}
char *dax_get_attr(dax_state *ds, char *name);
\end{verbatim}

This is a very simple function that takes the \textit{name} of the attribute
that you want and returns a pointer to the string.  This string is allocated
within the \verb|dax_state| object and should not be modified.  If your module
needs to store this string for later use you should make a copy of
it\footnote{The \texttt{strdup()} function works well for this}.  The pointer
will point to invalid information after the configuration has been freed.  We'll
discuss freeing the configuration shortly.

\subsection{Setting Attributes}

Setting attributes manually gives the module programmer more power over how
the configuration gets done.  One example might be that your module would
like to just receive the name of a configuration file on the command line.
It could read it's own arguments to find that filename, pass it to \verb|config|
and then run the configuration with only the configuration file.
The \verb|dax_set_attr()|\index{dax\_set\_attr() function} function is used 
to set those attribute values outside of the configuration system.

\begin{verbatim}
int dax_set_attr(dax_state *ds, char *name, char *value);
\end{verbatim}

The prototype should be pretty self explanatory.  The \textit{name} argument
should point to the name of the attribute you want to set and \textit{value}
should point to the value that you want the attribute to take.

It is important to note that any callbacks that are associated with this
attribute will be called as well.  This may have some usefulness.  Most modules
will not need to set attribute values.

\subsection{Finishing Up}

Once we are done with the configuration we can use the
\verb|dax_free_config()|\index{dax\_free\_config() function} function to free
the configuration memory.

\begin{verbatim}
int dax_free_config(dax_state *ds)
\end{verbatim}

This function simply takes a pointer to the \verb|dax_state| object and frees
the data associated with the configuration.  There are a lot of strings that are
maintained by the configuration system in the library and this is nothing more
than a way to free up that memory.  If your module will need access to these
configuration options and you don't want to make copies then you do not need to
call this function.

After calling this function any strings that you received from
\verb|dax_get_attr()| will no longer be valid, so you don't want to reference
those pointers any longer.

\section{Module Connection}
Before we can communicate to the server we must create a connection.

There are two functions that deal with module registration.
\begin{verbatim}
int dax_connect(dax_state *ds)
int dax_disconnect(dax_state *ds)
\end{verbatim}

The \verb|dax_connect()| function\index{dax\_connect() function} makes the
initial connection to the server, identifies the module to the server and takes
care of any other initialization issues that need to be handled.  Once the
module has successfully connected it can begin doing it's job.

It may be important to note that the \verb|dax_connect()| function creates
and runs a thread that will always be in the background handling the
communications to the server.

\verb|dax_connect()| returns 0 on success and an error code on failure.

The \verb|dax_disconnect()| function\index{dax\_disconnect() function} informs
the server that we are through and closes the connection.

The \verb|dax_connect()| can be called again after the module has been
disconnected.  If the module determines that there are communication errors it
could re-establish communications in this way.


\chapter{Dealing with Data}
The basic unit of data in OpenDAX is the \textit{Tag}.  A tag is similar to a variable in a programming language.
\section{Data Types}
Tags can be one of 15 different data types in OpenDax.  These are given in the following table.

\begin{tabular}{|l|l|l|l|l|}
\hline \textbf{Name} & \textbf{Description} & \textbf{Size} & \textbf{Min} & \textbf{Max} \\
\hline BOOL & Boolean (True/False) & 1 & 0 & 1 \\
\hline BYTE & Bit String & 8 & 0 & 255 \\
\hline SINT & Signed Short Integer & 8 & -128 & 127 \\
\hline CHAR & Character & 8 & -128 & 127 \\
\hline WORD & Bit String & 16 & 0 & 65,535 \\
\hline INT & Signed Integer & 16 & -32,768 & 32,767 \\
\hline UINT & Unsigned Integer & 16 & 0 & 65,535 \\
\hline DWORD & Bit String & 32 & 0 & $2^{32}$ \\
\hline DINT & Double Integer & 32 & $-2^{31}$ & $2^{31}-1$ \\
\hline UDINT & Unsigned Double Integer & 32 & 0 & $2^{32}$ \\
\hline REAL & IEC 754 Floating Point & 32 &  &  \\
\hline LWORD & Bit String & 64 & 0 & $2^{64}$ \\
\hline LINT & Long Integer & 64 & $-2^{63}$ & $2^{63}-1$ \\
\hline ULINT & Unsigned Long Integer & 64 & 0 & $2^{64}$ \\
\hline TIME & Unix Timestamp in mSec & 64 &  &  \\
\hline LREAL & IEC 754 Double Floating Point & 64 &  &  \\
\hline
\end{tabular}

The opendax.h header file contains the definitions for these data types.  These definitions are the name of the data type in the above table prefixed with "\texttt{DAX\_}".  So to represent a Boolean data type to the OpenDAX library you would use the definition \texttt{DAX\_BOOL}.  These definitions are used anytime your module needs to communicate a data type to the OpenDAX library.  For example, when creating a tag the module has to specify the data type.  You could create an INT tag with a call like this...

\begin{verbatim}
dax_add_tag("MyInt", DAX_INT, 1);
\end{verbatim}

This would create a tag in the server with the name \textit{MyInt} of type INT.  The 1 as the last argument just means a single member.  A larger number would signify an array.

There are also some typedefs in opendax.h that help with declaring variables within you module code.  They are the same as the datatype definitions except they are all lower case.  For example, if you want to declare a variable that will match the \verb|DAX_DINT| data type you would use \verb|dax_dint| typedef.  These make sure that the variable definitions inside your module match the data types of the tags in OpenDAX.  For the above Tag you could create a variable in C with this code...

\begin{verbatim}
dax_int myInt;
myInt = 13000;
\end{verbatim}

Each Tag in OpenDAX can be a single value of any of these base data types or it can be an array of these.

\begin{verbatim}
dax_add_tag("MyInt", DAX_INT, 10);
\end{verbatim}

This code would generate an array of INT's in the Server.  It is important to note that tags can be redefined as long as the data type stays the same.  If you call \verb|dax\_add\_tag()| again with a count that is higher than the previous call it would increase the size of the array.  If you call it with a count smaller than the previous, it will ignore it and keep the array the same size.  If you change the data type in the second call the function will return an error.

OpenDAX also supports the concept of a \textit{Compound Data Type}\index{Compound Data Type}.  This is an aggregate data type that is very similar to a structure in C.  Your module can create a new compound data type or it can use ones that are created by other modules.

These are the four functions that we will need to use from the libdax library.

\begin{verbatim}
dax_cdt *dax_cdt_new(char *name, int *error);

int dax_cdt_member(dax_state *ds, dax_cdt *cdt, char *name,
                   tag_type mem_type, unsigned int count);

int dax_cdt_create(dax_state *ds, dax_cdt *cdt, tag_type *type);

void dax_cdt_free(dax_cdt *cdt);
\end{verbatim}

To create a CDT you first have to allocate a CDT object.

The \verb|dax_cdt_new()|\index{dax\_cdt\_new() function} allocates, initializes and returns a pointer to a new CDT object.  The \textit{name} argument is the name that would be given to the CDT.  The \textit{error} argument is a pointer to an integer that can indicate any errors.  If there is an error the function will return NULL and the integer pointed to \textit{error} will be set to the error code.  If you are not interested in this error code then you can pass NULL to the function.

Once you have the object you add members to the CDT one at a time.  These members can be of any previously defined data type including other CDTs.  They can also be arrays and even arrays of other CDT's.  We do this with the \verb|dax_cdt_member()|\index{dax\_cdt\_member() function}.  The \textit{cdt} argument is the object that was returned from the \verb|dax_cdt_new()| function.  The \textit{name} argument is the name that we want to give to our member.  The \textit{mem\_type} argument is either one of the 15 base DAX\_* datatypes from above or a predefined compound data type.

Once all of the members have been added the module the \verb|dax_cdt_create()|\index{dax\_cdt\_create() function} function is used to send the data type definition to the server and actually create the data type.  The \textit{type} argument to \verb|dax_cdt_create()| will contain the new type identifier of the created compound data type.  This identifier can be used anywhere another data type definition (such as DAX\_DINT) could be used.

The \verb|dax_cdt_free()|\index{dax\_cdt\_free() function} function simply
frees the memory associated with the new data type.  If the
\verb|dax_cdt_crete()| function is used it will free the object so this
function is only necessary when errors happen and the construction of the
data type is being aborted.  Do Not try to free the memory yourself because
there are other data structures in the \verb|dax_cdt| object that have to be
freed.  Simply passing the pointer to \verb|free()| for example will result
in a memory leak.  The \verb|dax_cdt| object should
not be reused.  You can reuse the pointer but old object needs to be
freed with \verb|dax_cdt_free()| or \verb|dax_cdt_crete()| first, and then
reallocate a new one with  \verb|dax_cdt_new()|.

Perhaps it's time for an example.  Let's build a CDT that has the following structure...

\begin{verbatim}
MyCDT
  MyInteger INT
  MyReals   REAL[10]
  MyBools   BOOL[8]
\end{verbatim}

The following code would be used \ldots

\begin{verbatim}
dax_cdt *dc;
int error;
tag_type type;

dc = dax_new("MyCDT", &error);
if(dc == NULL) {
    printf("Error code returned is %d\n", error);
    //handle error
} else {
    dax_cdt_member(ds, dc, "MyInteger", DAX_INT, 1);
    dax_cdt_member(ds, dc, "MyReals", DAX_REAL, 10);
    dax_cdt_member(ds, dc, "MyBools", DAX_BOOL, 8);

    error = dax_cdt_create(ds, dc, &type);
    if(error) {
        dax_cdt_free(dc);
        printf("Error from dax_cdt_member() is %d\n", error);
	    //handle error
    } else {
        printf("CDT created!  Data Type = 0x%X\n", type);
    }
}

\end{verbatim}

This code will build the above CDT.  There is some error checking missing on the return values of \verb|dax_cdt_member()| functions but this is for clarity.  In your code you would want to check these return values.  Also be careful of the error handling so that you don't exit this code before the \verb|dax_cdt_free()| function has been called to free the \verb|dax_cdt| object.

% NEED TO ADD A SECTION ON GETTING THE TAG'S SIZE

% NEED TO ADD A SECTION ON HOW TO ITERATE THROUGH A CDT

\section{Creating Tags}
The main function to use for creating tags for \opendax is \verb| dax_tag_add()| function\index{dax\_tag\_add() function}.  The prototype is given here.

\begin{verbatim}
int dax_tag_add(dax_state *ds, tag_handle *h, char *name,
                tag_type type, int count)
\end{verbatim}

There are three pieces of information that must be passed to this function to create a tag.  First you have to pass a \emph{name}.  The name is statically allocated so the size of the tag is fixed.  It must be less than \verb|DAX_TAGNAME_SIZE|.  Right now this is set at 32 characters.  It can easily be changed to allow longer tag names or made smaller to conserve memory in smaller systems.  The definition is in \emph{opendax.h} and once changed you will have to recompile everything.

The rules for naming tags are pretty simple.  The tag must begin an alphabetic character [a-z or A-Z].  The rest of the tag name can contain underscores, letters or numbers.  For example \verb|ThisTag| and \verb|some_tag23| are both valid tag names.  However \verb|23some_tag| is not a valid tag name.  Hyphens and symbols are not allowed.  Tags that begin with "\_" are reserved for the system's use.

The \emph{type} argument, is the data type of the tag.  This can be one of the base data types or a compound data type.

The  last argument, \emph{count}, is the number of items in the tag.  Basically the tag is either a single tag and \emph{count} would be 1 or the tag represents an array and \emph{count} would be something greater than 1.

The \emph{h} pointer argument is a pointer to a handle\index{handle}.  If this pointer is not NULL \verb|dax_tag_add()| will fill it in with the necessary information to identify the tag.  Tag handles are described in detail in the following section.

\section{Tag Handles}
It would be very inefficient for the server/library to have to work with the strings that represent different tags in this system.  Strings would use a lot of bandwidth and parsing the strings to determine the actual data point would use extra processor power.

To alleviate some of these concerns \opendax uses the idea of a \emph{handle} to identify tags.  They can also be used to identify parts of the tags.  For example, let's say that we have a tag that is an array of 1,000 INT's named \emph{MyInt}.  The handle that identifies the entire array would be different than one that would identify the items from MyInt[100] through MyInt[199], and that would be different than any of the 1,000 handles that would identify a single point in the array.

A \verb|tag_handle| is actually a structure that is defined in \emph{opendax.h}.

There are a couple of ways to get a handle assigned.  The first is to pass a pointer to the \verb|dax_tag_add()| function.  \verb|dax_tag_add()| will assign the information necessary to access the entire tag when the tag is created.

\begin{verbatim}
int dax_tag_handle(dax_state *ds, tag_handle *h,
                   char *str, int count)
\end{verbatim}

The \verb|dax_tag_handle()| function \index{dax\_tag\_handle() function} is the other way to get a handle.  The arguments to this function are a pointer to the handle that you want to have assigned, a string that represents the tag for which you would like the handle and an integer that represents the number of items you would like assigned to the handle in the case that the tag is an array.

The string passed to \verb|dax_tag_handle()| can be more than simply the tag name.  It can be a full description of part of a tag.  Suppose we are using the compound data type that we defined earlier.

\begin{verbatim}
MyCDT
  MyInteger INT
  MyReals   REAL[10]
  MyBools   BOOL[8]
\end{verbatim}

Let's say we create a tag that is a 10 item array of this type, named \emph{MyTag}.  We could represent parts of this tag with the following strings \ldots

\begin{tabular}{lll}
\textbf{str} & \textbf{cnt} & \textbf{Handle represents \ldots} \\
\verb|"MyTag"| & 1 & the entire tag \\
\verb|"MyTag[0]"| & 1 & the entire first element \\
\verb|"MyTag[0]"| & 2 & all of the first two elements \\
\verb|"MyTag[1].MyInteger"| & n/a & the MyInteger member of the 2nd element \\
\verb|"MyTag[1].MyBools"| & n/a & the entire MyBools array of the 2nd element \\
\verb|"MyTag[2].MyReals[1]"| & 2 & two MyReals starting with the 2nd one \\
\end{tabular}

This is the way that tags and parts of tags should be represented within your module as well.  This representation should be pretty familiar to a C programmer.  Square brackets and numbers are used to index the arrays and '.'s are used to dereference parts of a tag that have a compound data type.  Arrays are zero indexed.

If count is zero and the tag or member is an array the entire array will be assigned to the handle.

Here is the definition of the tag\_handle structure.

\begin{verbatim}
typedef struct tag_handle {
	tag_index index;     /* The Database Index of the Tag */
	uint32_t byte;       /* The byte offset where the data block starts */
	unsigned char bit;   /* The bit offset */
	uint32_t count;      /* The number of items represented by the handle */
	uint32_t size;       /* The total size of the data block in bytes */
	tag_type type;       /* The data type of the block */
} tag_handle;
\end{verbatim}

The tagserver is actually pretty dumb when it comes to retrieving data from the tag database.  The
tagserver only needs the index, the byte offset and the number of bytes.  It's really up to the
module to figure out what to do with that data.  The libdax library hides most of this complexity
from the module programmer.

\section{Reading and Writing Data}

Once you have a handle to the data you can start to read and write that data.  The three functions that you will use to read and write data are \ldots

\begin{verbatim}
int dax_read_tag(dax_state *ds, Handle handle, void *data);
int dax_write_tag(dax_state *ds, Handle handle, void *data);
int dax_mask_tag(dax_state *ds, Handle handle, void *data,
                 void *mask);
\end{verbatim}

The \verb|dax_read_tag()| function \index{dax\_read\_tag() function}, as its
name suggests is used to read tag data from the server.  The \verb|void
*data| pointer should point to data that is formatted in the same way that
the handle data is.  Usually this simply means that the variables
in you module should be of the same type and size as what is pointed to by
\emph{handle}.

The \verb|dax_write_tag()| function \index{dax\_write\_tag() function} and
the \verb|dax_mask_tag()| function \index{dax\_mask\_tag() function} are both
used to write data to the server.  Again the \emph{data} pointer should point
to a data area within your module that is of the same type and size as the
tag that is represented by the \emph{handle}.  The \verb|dax_mask_tag()|
function function adds the ability to only write part of the data.  The
\emph{mask} pointer should point to an area of memory that is the same size
as the \emph{data} pointer.  The only \emph{data} that will be written to the
server will have a '1' set in that bit location in the \emph{mask} memory
area.

It might be important to understand how large tags are read.  If
the size of the request is larger than the maximum number of bytes that
can be read in a single message, the request will be split up.
This causes a race condition because another module could write to the tags
between these messages.  This will likely, not be an issue for the vast
majority of cases, but it is worth knowing.

Writing tags is also split up into multiple messages if needed.  The same
race condition exists here as well.  Also the amount of data that can be
written is slightly less than can be read.  This is due to the fact that
the message header is larger for writes.
Masked writes are half the size of normal writes, since room has to be made
for the mask.

\section{Compound Data}

See the implementation of the \verb|dax_cdt_iter()| function \index{dax\_cdt\_iter() function} to see how to deal with compound data.  The easiest and least efficient way is to get handles to the individual members of the CDT.

The packing of the CDT is also specified and it's pretty simple.  The members
are packed in the same order in which they were added by the
\verb|dax_cdt_member()| function \index{dax\_cdt\_member() function}.  They
are single byte aligned and BOOLs are packed into bytes.  One BOOL will take
up an entire byte.  Any subsequent BOOL's that are added will occupy that
same byte until there are more than 8 in which case another byte will be
added.  This continues until the first non-BOOL data type.  The first
non-BOOL typed member will occupy the next byte.  For this reason it is more
efficient to put the BOOL members together.  A single BOOL member followed by
an INT then followed by another BOOL will occupy 4 bytes, one for the first
BOOL, two for the INT and then another byte for the second BOOL.  If we put
the two BOOL members first and second, with the INT last, the data type size
will only be 3 bytes, and we still have room for 6 more BOOLs before the size
would grow.

\section{Data Handling Example}




\chapter{Handling Events}
Event Notification is a feature of OpenDAX that allows a module to be notified immediately when a newly written tag meets some criteria.  For instance you may have an I/O module that wants to know the instant that some tag changes so that you can update the hardware outputs with that tag data.

There are 8 types of events that can be assigned to a tag.  Figure \ref{Event Matrix} depicts which event types are allowed for each data type.  Event types are allowed on arrays if they are allowed on the base types.  The event can be defined over a subset of the range of the tag.  Let's say that you have an array of Integers named \textit{MyInts} you only want to know if the 3rd, 4th or 5th values become greater than 12.  You would add an event for \textit{MyInts[2]} with a count of 3, a type of \textit{Greater Than} and the number 12.  Then the server will notify your module when any of those values became greater than 12.

A \textit{Write} event happens anytime the tag is written to weather the data is different or not.  The server does not check the data that is written just that the range of the write overlaps with the range that the event is associated with.

The \textit{Change} event will fire anytime the tag data changes.

The \textit{Write} and \textit{Change} events are the only two events that will work with every type of data.

The \textit{Set} and \textit{Reset} events happen when a Boolean is Set or Reset, respectively.  It will only fire when the bit makes the transition.  In the case of the \textit{Set} event the module will be notified only when the bit makes the transition from 0 to 1.  After that the event will not fire no matter how many times the bit is written to 1.  The \textit{Reset} event is the same way except that it will only fire when the bit transitions from 1 to 0.

If the event corresponds to an array of bits it will only fire once per write operation.  So if a single write operation results in multiple bits being set (or reset) there will be only one event sent to the module.

The \textit{Equal} event fires when a tag's value equals the value stored in the events \verb|data| parameter.

The \textit{Greater Than} and \textit{Less Than} events are just what you would expect.  They fire each time the new tag value is greater than or less than the event's \verb|data| parameter.

The \textit{Deadband} event fires when the data has changed by the amount given in the \verb|data| parameter.  For instance if the tag's value is 4 and the data value is 2 the event will not fire if the tag value is written as 3 or 5 but it will fire when the tag's value becomes 2 or 6.  To put it mathematically the event fires once the absolute value of the tag's old value minus the tag's new value is greater than or equal to data.

The \textit{Equal}, \textit{Greater Than}, \textit{Less Than} and \textit{Deadband} events all have a \verb|data| parameter associated with them.  This is a single data parameter regardless of how many items within the tag the event is associated with.  For example, if you have an array of DINT's and you would like to know if any one of them were to become greater than 7 you would create a \textit{Greater Than} event with data = 7 for the entire array.  The event will fire once per write cycle so even if more than one item in the array exceeds 7 the module will only be notified once.  If you would like to be notified when some items exceed 7 but others exceed 9 then you will have to create multiple events to handle this.

Also, the \textit{Equal}, \textit{Greater Than} and \textit{Less Than} events will only fire once when they are satisfied.  They must become untrue and then true again before another event will be fired.

The \textit{Deadband} event will reset the "old value" when it fires.  So in the above example when the value is 4 and the deadband data is 2 the event will fire when the value becomes 6 but it won't fire again until the value changes by 2 from that 6.  Now the value must either fall back to 4 or become 8 to get this event again.
2
\begin{figure}[h]
\centering
\begin{tabular}{|l|c|c|c|c|c|c|c|c|}
\hline   & Write & Change & Set & Reset & = & $<$ & $>$ & Deadband \\
\hline  BOOL & $\bullet$ & $\bullet$ & $\bullet$ & $\bullet$ &  &  &  &  \\
\hline  BYTE & $\bullet$ & $\bullet$ &  &  & $\bullet$ & $\bullet$ & $\bullet$ & $\bullet$ \\
\hline  SINT & $\bullet$ & $\bullet$ &  &  & $\bullet$ & $\bullet$ & $\bullet$ &$\bullet$  \\
\hline  WORD & $\bullet$ & $\bullet$ &  &  & $\bullet$ & $\bullet$ & $\bullet$ & $\bullet$ \\
\hline  INT & $\bullet$ & $\bullet$ &  &  & $\bullet$ & $\bullet$ & $\bullet$ & $\bullet$ \\
\hline  UINT & $\bullet$ & $\bullet$ &  &  & $\bullet$ & $\bullet$ & $\bullet$ & $\bullet$ \\
\hline  DWORD & $\bullet$ & $\bullet$ &  &  & $\bullet$ & $\bullet$ & $\bullet$ & $\bullet$ \\
\hline  DINT & $\bullet$ & $\bullet$ &  &  & $\bullet$ & $\bullet$ & $\bullet$ & $\bullet$ \\
\hline  UDINT & $\bullet$ & $\bullet$ &  &  & $\bullet$ & $\bullet$ & $\bullet$ & $\bullet$ \\
\hline  TIME & $\bullet$ & $\bullet$ &  &  & $\bullet$ & $\bullet$ & $\bullet$ & $\bullet$ \\
\hline  LWORD & $\bullet$ & $\bullet$ &  &  & $\bullet$ & $\bullet$ & $\bullet$ & $\bullet$ \\
\hline  LINT & $\bullet$ & $\bullet$ &  &  & $\bullet$ & $\bullet$ & $\bullet$ & $\bullet$ \\
\hline  ULINT & $\bullet$ & $\bullet$ &  &  & $\bullet$ & $\bullet$ & $\bullet$ & $\bullet$ \\
\hline  REAL & $\bullet$ & $\bullet$ &  &  &  & $\bullet$ & $\bullet$ & $\bullet$ \\
\hline  LREAL & $\bullet$ & $\bullet$ &  &  &  & $\bullet$ & $\bullet$ & $\bullet$ \\
\hline  Compound & $\bullet$ & $\bullet$ &  &  &  &  &  &  \\
\hline
\end{tabular}
\caption{\label{Event Matrix}Event Matrix}
\end{figure}

The functions that we will need are...

\begin{verbatim}
int dax_event_add(dax_state *ds, tag_handle *handle, int event_type,
                  void *data, dax_event_id *id,
                  void (*callback)(void *udata), void *udata);
\end{verbatim}
\index{dax\_event\_add() function}
\begin{verbatim}
int dax_event_del(dax_state *ds, dax_event_id id);
\end{verbatim}
\index{dax\_event\_del() function}
\begin{verbatim}
int dax_event_select(dax_state *ds, int timeout, dax_event_id *id);
\end{verbatim}
\index{dax\_event\_select() function}
\begin{verbatim}
int dax_event_poll(dax_state *ds, dax_event_id *id);
\end{verbatim}
\index{dax\_event\_poll() function}
\begin{verbatim}
int dax_event_get_fd(dax_state *ds);
\end{verbatim}
\index{dax\_event\_get\_fd() function}
\begin{verbatim}
int dax_event_dispatch(dax_state *ds, dax_event_id *id);
\end{verbatim}
\index{dax\_event\_dispatch() function}

\section{Adding Events}

The \eventadd\index{dax\_event\_add() function} function is used to add a new event to the server.  It is a pretty involved function with a lot of arguments but it is actually pretty simple.  The first argument is the \daxstate| that we are very familiar with, the second argument is a pointer to a handle that represents the tag that we would like the event to be associated with.

\begin{verbatim}
int dax_event_add(dax_state *ds, tag_handle *handle, int event_type,
                  void *data, dax_event_id *id,
                  void (*callback)(void *udata), void *udata);
\end{verbatim}

The \texttt{event\_type} argument is one of the following...

\begin{verbatim}
EVENT_WRITE
EVENT_CHANGE
EVENT_SET
EVENT_RESET
EVENT_EQUAL
EVENT_GREATER
EVENT_LESS
EVENT_DEADBAND
\end{verbatim}

These are defined in \textit{opendax.h} and represent each of the events, and should be obvious which events they represent.  The \verb|void *data| parameter is a void pointer to a data point that matches the datatype of the handle.  It is only needed for \verb|EVENT_EQUAL|, \verb|EVENT_GREATER|, \verb|EVENT_LESS| and \verb|EVENT_DEADBAND|.  It will be ignored for the rest and can be set to NULL.  It is important that the programmer make sure that this pointer points to the correct type of data.  If the data does not match the datatype of the handle there might be trouble.  The library has no way to verify that the programmer has done this correctly.

The next argument, \verb|dax_event_id *id| is a pointer to a \texttt{dax\_event\_id}\index{dax\_event\_id structure} structure.  This structure is defined in \textit{opendax.h} and contains the data that would be necessary to uniquely identify this particular event to the server.  If this argument is set to NULL then nothing will be set here.  If you ever want to modify or delete this event you will have to have this identifier.

The \texttt{callback} argument is the callback function that you would like to have the event dispatching system call when this event is received by your module.  The prototype for the function should be...
\begin{verbatim}
void function_name(void *udata)
\end{verbatim}
The \texttt{udata} that the event dispatcher will pass is the same pointer you supply as the last argument to \eventadd.  The system doesn't look at the data, doesn't care about the data, will not free the data when the event is deleted or do anything other than simply pass this same pointer to your callback function.  If you have one callback function per event you may not need it but if you have a single callback function that handles multiple events you will need to have the event dispatcher pass some information to that callback function, that it can use to differentiate between the different events.

I suspect it will be very common for the \texttt{udata} parameter to be a pointer to the same handle of the tag that the event is associated with.  The callback function will most likely then go to the server read the data from that handle and do whatever it is that the module does.

Once you are finished with an event you can delete it with \eventdel.

\begin{verbatim}
int dax_event_del(dax_state *ds, dax_event_id id);
\end{verbatim}
\index{dax\_event\_del() function}

The second argument is simply the \texttt{id} structure that was returned by \eventadd.  The \eventdel function will not free the data pointed to by \texttt{udata} so you must be careful to take care of that detail yourself.

\section{Handling Events}

There are basically three ways to receive and handle events.  The first way is with the \eventwait function.  This function blocks and waits for the event to happen.  The second way is with the \verb|dax_event_poll()|\index{dax\_event\_poll() function} function, that checks for an event and immediately returns whether it deals with the event or not.  The last way to deal with events is to get the file descriptor of the socket that is being used to receive events and handle them yourself.

\begin{verbatim}
int dax_event_wait(dax_state *ds, int timeout, dax_event_id *id);
int dax_event_poll(dax_state *ds, dax_event_id *id);
\end{verbatim}
\index{dax\_event\_wait() function}
\index{dax\_event\_poll() function}


The \texttt{dax\_event\_wait()} function takes a couple of arguments in addition to the normal \daxstate argument.  The first is \texttt{timeout}.  This is how long you would like to wait for the event before the function returns.  If the function returns a \verb|0| then it dispatched and event.  If it returns \texttt{ERR\_TIMEOUT} then it did not dispatch an event and has timed out.  This gives you the ability to block waiting for events but occasionally exit and handle some other details.  \eventwait will only handle one event and then return.  Even if there are more events in the buffer to deal with.

The last parameter to \eventwait is a pointer to a \texttt{dax\_event\_id} structure.  This will be filled in by \eventwait with the information that identifies this event.

If you passed a pointer to a callback function to the \eventadd function, \verb|dax_event_wait()| will call that function and pass the \texttt{udata} pointer that you passed to \eventadd when you created the event.

The  function is very similar to \verb|dax_event_wait()| except that it will not wait.  It returns immediately if there is no event to dispatch.  If it dispatches an event it returns 0 and \verb|ERR_NOTFOUND| if there was no event.  Both functions might return another error code if there was some other problem.

\begin{verbatim}
int dax_event_poll(dax_state *ds, dax_event_id *id);
\end{verbatim}
\index{dax\_event\_poll() function}

The final method of dealing with events allows your program to have complete control over the file descriptor.  This is more complex, but it might be necessary if your program needs to deal with other events.  First use \texttt{dax\_event\_get\_fd()}\index{dax\_event\_get\_fd() function} function to retrieve the file descriptor that the library is using to wait on events to come from the server.  You can use other system calls such as \verb|system()| or \verb|poll()| to determine if there is a read pending.  If there is any information to read your module would then call \texttt{int dax\_event\_dispatch(dax\_state *ds, dax\_event\_id *id)}\index{dax\_event\_dispatch() function} function.  This function carries out the same tasks as \texttt{dax\_event\_wait()} and \texttt{dax\_event\_poll()}.  In fact these two functions call \texttt{dax\_event\_dispatch()} from within the library.

It is important that your module not read any of the data from the socket that the file descriptor represents.  Also, the data is one way so you only need to detect that the socket is ready for reading.

\chapter{Lua Modules}

It is possible to write an OpenDAX module entirely in Lua.  Included in the OpenDAX distribution is a Lua module for interfacing to the OpenDAX API from Lua. 

See the Lua API documentation in the appendix for details.

\section{Using the Lua Interface in C Modules}

<<<Needs Writing>>>

\appendix

\chapter{Lua API Reference}
\section{Introduction}
The Lua API is a wrapper around the \textit{libdax} library for use in Lua scripts.  Lua scripts can either be run by the \textit{daxlua} module or they can run as stand alone Lua scripts.

If the script is run from the \textit{daxlua} the functions will have bee loaded and the connection to the tag server will have already been made and initialized.

If the script is run as a stand alone interpreter then the \textit{dax} module will have to be loaded and the connection to the server initalized.

\section{General}

\begin{verbatim}
dax.init(modulename)
\end{verbatim}
\index{dax.init function}
Initialize the internal library and attempt to connect to the tagserver.  The function takes one arguement, \textit{modulename}, which is a string that represents the name that will be used to register this module with the server.

This function is only needed for stand alone Lua client modules.  For scripts that are run within the \textit{daxlua} module this initialization has already been done.

This function returns nothing and will raise errors if something fails.
\begin{verbatim}
dax.free()
\end{verbatim}
\index{dax.free function}
Disconnects from the server and frees the internal library data structures.

This function is only needed for stand alone Lua client modules.  For scripts that are run within the \textit{daxlua} module this initialization has already been done.

This function returns nothing and should never fail.
\begin{verbatim}
dax.cdt_create(typename, cdt_table)
\end{verbatim}
\index{dax.cdt\_create function}
The function is used to create a \textit{Compound Datatype}.  The first argument should be a string that will be used as the name of the CDT.  The second argument is a table of tables that defines the members of the CDT.

\begin{verbatim}
members = {{"Name", "DataType", count},
           {"AnotherNmae", "DataType", count}}
\end{verbatim}

This function raises erros on failure and returns a single integer that represents the datatype and can be used to create tags.

\begin{verbatim}
dax.tag_add(name, type, <count>)
\end{verbatim}
\index{dax.tag\_add function}

Adds a tag to the tagserver database.  The first argument is a string that represents the name of the new tag.  The second argument can either be an integer or a string that represents the data type of the tag.  The third argument is represents the number of items created for the datatype.  If this number is greater than 1 then an array is created.  If the count is not given then 1 is assumed.

This function returns nothing on success and raises errors otherwise.

\begin{verbatim}
dax.tag_get(tag)
\end{verbatim}
\index{dax.tag\_get function}
Retrieve the definition of the given tag.  The function takes a single argument that can either be the tagname as a string or the tag index as an integer.

The function returns three values that represent the tag.  They are name, type and count.

\begin{verbatim}
dax.tag_read(tag, <count>)
\end{verbatim}
\index{dax.tag\_write function}
Read and return the value(s) of the given tag.  The first argument is a string representing the tag that we wish to read.  The second, optional, argument is the number of members that we want to read.

The return value depends the type and size of the tag.

\begin{verbatim}
dax.event_add(tagname, count, type, data, callback, userdata)
\end{verbatim}
\index{dax.event\_add function}
Add an event to the system.  The function takes six arguments.

tagname - The first argument is the tagname of the tag that we wish to monitor with this event.

count - The number of items in the tag we wish to monitor.

type - A string representing the event type.  i.e. WRITE, CHANGE, LESS

data - A number that is used for the comparison events EQUAL, GREATER and LESS

callback - The function that we wish to be called when the event is triggered

userdata - This data will be stored by the system and passed to the callback function when this event is triggered

On sucess the function returns a table that represents this event.  On failure errors are raised.

\begin{verbatim}
dax.event_del(event)
\end{verbatim}
\index{dax.event\_del function}
Removes the event from the system.  The single argument to this function is the table that was returned by the \textit{event\_add} function.  

This function returns nothing on success and raises errors on failure.

\begin{verbatim}
dax.event_wait(<timeout>)
\end{verbatim}
\index{dax.event\_wait function}
Blocks and waits for any event to happen and then dispatches that event.  Timeout represents how long the fuction will wait for an event.  If timeout is not given or is set to 0 the function will block forever.

Only a single event will be handled by this function.

The function returns 0 on timeout and 1 if an event was dispatched.

\begin{verbatim}
dax.event_poll()
\end{verbatim}
\index{dax.event\_poll function}
Similar to \textit{event\_wait} except that it does not wait.  If there is an event that is ready to be handled then it will be handled and 1 is returned.  Otherwise it returns 0 immediately.

\chapter{Skeleton Module}
The Skeleton module is included in the distribution as a starting place for building new modules.  It includes just enough code to function as an OpenDAX module.  The code is well documented and when used with this manual it will, hopefully, make writing a new module for OpenDAX a fairly simple and straight forward process.  The Skeleton module is located in the /modules/skel subdirectory of the distribution.

\section{skel.h}
\begin{tiny}
\verbatiminput{../src/modules/skel/skel.h}
\end{tiny}
\section{skel.c}
\begin{tiny}
\verbatiminput{../src/modules/skel/skel.c}
\end{tiny}

%\input{dev_codebase.tex}
%\chapter{opendax.h Listing}
%\begin{tiny}
%\verbatiminput{../src/opendax.h}
%\end{tiny}

%\chapter{GPL License}
%\verbatiminput{../COPYING}

\printindex
\end{document}
