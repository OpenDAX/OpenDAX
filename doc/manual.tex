\documentclass[10pt,letterpaper]{report}

\usepackage[american]{babel}
\usepackage{amsmath}
\usepackage{amsfonts}
\usepackage{amssymb}
\usepackage{setspace}
\usepackage{parskip}
\usepackage[dvips]{graphicx}
\usepackage{float}
\usepackage{makeidx}
\usepackage{verbatim}
\usepackage{color}
\usepackage{hyperref}
\hypersetup{
    colorlinks=true, % make the links colored
    linkcolor=blue, % color TOC links in blue
    urlcolor=red, % color URLs in red
    linktoc=all % 'all' will create links for everything in the TOC
}

\setlength\parskip{10pt}
\setcounter{secnumdepth}{3}

\makeatletter
\def\thickhrulefill{\leavevmode \leaders \hrule height 1pt\hfill \kern \z@}
\renewcommand{\maketitle}{\begin{titlepage}%
    \let\footnotesize\small
    \let\footnoterule\relax
    \parindent \z@
    \reset@font
    \null\vfil
    \begin{flushleft}
      \huge \@title
    \end{flushleft}
    \par
    \hrule height 1pt
    \par
    \begin{flushright}
      \LARGE \@author \par
    \end{flushright}
    \vskip 60\p@
    \vfil\null
  \end{titlepage}%
  \setcounter{footnote}{0}%
}

\makeatother
\makeindex

\def\opendax{\textit{OpenDAX}}
\def\modbus{\textit{Modbus}$\textsuperscript{\textregistered}$}
\def\daxstate{\texttt{dax\_state} }
\def\eventadd{\texttt{dax\_event\_add()}}
\def\eventdel{\texttt{dax\_event\_del()}}
\def\eventwait{\texttt{dax\_event\_wait()}}
\def\eventpoll{\texttt{dax\_event\_poll()}}
\def\eventgetfd{\texttt{dax\_event\_get\_fd()}}
\def\eventdispatch{\texttt{dax\_event\_dispatch()}}


\title{OpenDAX User's Manual}
\date{June 13, 2008}
\author{Phil Birkelbach}


\begin{document}
	\pagenumbering{roman}
	\maketitle

	\begin{flushleft}
		Copyright \textcopyright 2008 - Phil Birkelbach\linebreak
		All Rights Reserved
	\end{flushleft}

	\tableofcontents
	\newpage
	\pagenumbering{arabic}
	\chapter{Introduction}
	\opendax{} is an open source, modular, data acquisition and control system
	framework. It is licensed under the GPL (GNU Public License) and therefore is
	completely free to use and modify. \opendax{} is written primarily for Linux.
	There should be no reason that it wouldn't compile on other *nix like operating
	systems but for now we are concentrating on Linux.

	\opendax{} could be used for anything from controlling the air conditioner in a
	home to controlling an entire industrial facility. Depending on what modules are
	installed and run it could be used as the firmware for a dedicated Programable
	Logic Controller (PLC) or a Home Automation system. It could loosely be compared
	to DCS (Distributed Control System) or a SCADA (Supervisory Control and Data
	Acquisition) system. Eventually the system would be able to scale up to a
	several hundred thousand tag system. At this moment the code is far to immature
	to be used for anything that required reliability but we hope to get it to that
	point sooner or later. Much will depend on how many developers take up the
	challenge to help work on the code.

	DAX stands for Data Acquisition and eXchange. The system works by combining a
	master program (opendax), a real time database server (tagserver) a library
	(libdax) and set of modules. The opendax server handles the starting, stopping
	and monitoring of process that make up a particular OpenDAX system.  It is
	actually an optional part of the system.  The user can start and stop each
	process manually if desired.

	The tag server is the program that manages the real time data that is used
	throughout the system.  It could hold the temperature or pressure of a process,
	the status of a switch or some command data from the Human Interface to a logic
	module.  Each piece of information is called a 'Tag'  Each tag can be a single
	value or an array of values.  (Multi dimensional arrays are not supported.)
	Each tag is defined as a given data type.  There are several base data types.
	These represent simple number types like INT or FLOAT.  It is possible to create
	compound
	data types that are collections of the base data types and other compound data
	types.

	The modules do all the work and communicate with the \texttt{tagserver} through
	the \texttt{libdax} library. There could be modules for reading and writing to
	I/O points, data logging, alarming, machine interface, and logic. The primary
	interface to the tag server is the \texttt{daxc} module, which is a command line
	interface module that can
	be used to read and write tags, check status and do all other types of
	maintenance in the tag server.

	There is no requirement that all of the modules have to be on the same machine.
	There is also no requirement that the modules be on the machine with the tag
	server.  Any process that is to be started by the \texttt{opendax} server will
	have to be on the same computer.  Modules communicate to the server by one of
	two mechanisms.  Either a TCP/IP connection or a \textit{local domain socket}.
	The \textit{local domain socket} has the advantage of being very fast because
	the operating system kernel simply copies memory from one process to another.
	The disadvantage is that the module that wants to use this communication
	mechanism must be on the same machine as the tag server. If the module is on
	another machine then it will have to use the TCP/IP protocol to communicate with
	the tag server.

	The system is designed so that most of the work happens in the client library
	and the modules.  Since the tag
	server is a bottleneck for the entire system, any work that could be offloaded
	to the client library and the
	modules was.

	\chapter{Getting Started}
	\section{Installation}

OpenDAX uses the CMake build system generator.  You'll need to install CMake
on your system.

You will also need the Lua development libraries installed.  Most 
distributions have versions of Lua that will work.  The currently supported
versions of Lua are 5.3 and greater.
If you install Lua from the source files you will need to add -FPIC 
compiler flag to the build.

\begin{verbatim}
make MYCFLAGS="-fPIC" linux
\end{verbatim}

Once you have CMake and the Lua libraries installed you can download and build
OpenDAX.  First clone the repository...

\begin{verbatim}
git clone https://github.com/OpenDAX/OpenDAX.git
\end{verbatim}

This should create the OpenDAX directory.  Now do the following...
\begin{verbatim}
mkdir build
cd build
cmake ..
make
make test
\end{verbatim}

If all the tests pass you can install with \ldots

\begin{verbatim}
sudo make install
sudo ldconfig
\end{verbatim}


	\section{Configuration}
	There are two ways to pass configuration information into any given \opendax
	program.  The first is simply to pass command line arguments and the second is
	via a configuration file.  There is likely some overlap between what options can
	be passed with either method but some options only make sense for the command
	line (an example is the \texttt{config} option that specifies which
	configuration file to use).  Some options might be too complex to put on the
	command line and will require a configuration file.

	If a configuration attribute is present on the command line it will take
	precedence over that same attribute given in the configuration files.  An
	exception to this is the logging configuration.  Once logging services are added
	via the configuration file those services will determine how messages are logged
	and anything passed on the  command line will be ignored.

	The \opendax configuration system is very sophisticated.  It is based on the
	\textit{Lua} scripting language.  In fact all of the configuration files in the
	base \opendax programs are nothing more than Lua scripts.  These scripts run in
	the sandbox of the program that calls them and some global data is pre-defined
	to help.

	Using a scripting language like Lua makes configuration very powerful.  Some of
	the simple things that you can do are using variables to store commonly used
	values, such as a file system path or an IP address that needs to be reused in
	many parts of the configuration.  If there are parts of the configuration that
	are common between many client modules, that information can be stored in single
	files and the function \texttt{dofile()} can be used to import that code into
	many configurations.  A common example of this is the message logging
	configuration.

	Before a configuration script is executed certain 'globals' are assigned to the
	script.  A very important one is the string \texttt{calling\_module}.  This
	string contains the name of the module that is running the configuration script.
	This allows for common code to be shared between different parts of the system
	and simple \texttt{if} statements to be used to separate which code will be run.
	The \texttt{print()} function is overridden in the configuration script to send
	the given string to the message logging facility instead of simply printing to
	the console.  These messages will be sent to any service that has the
	\texttt{CONFIG} topic configured.

	Other functions and globals may be set for different modules.  See the
	individual module's documentation for more information.

	The default configuration file for the tagserver is \texttt{./tagserver.conf}
	and the default configuration filename for the master daemon is
	\texttt{./opendax.conf}.  Typically the default configuration filename for a
	client module is that module's name with \texttt{.conf} extension.  This can
	overridden with the \texttt{-C} command line option.

	\chapter{The OpenDAX Master Server}

    The \opendax{} Master Server is a program that manages all of the
    other programs in the \opendax{} system.  It can be configured to
    start and monitor all of the process that are needed for any
    given application.

    The Master Server is named \texttt{opendax}.

    <<Need much more work here>>

	\chapter{The Tag Server}

    \section{Overview}
	The tag server contains the central database of all the real-time data.  The
	entire system is centered around the tag server.

	The tag server is named \texttt{tagserver} and it is very simple to run
	and will
	typically run just fine without any configuration using the defaults.  There are
	only a few configuration options.

    \section{Configuration}

	\begin{list}{}{Command Line Options}
		\item[\texttt{--configfile, -C}] - This option requires an argument that
		represents the configuration file that will be read on system start.  By default
		the file \texttt{./tagserver.conf} will be used.
		\item[\texttt{--version, -V}] - If this option is set, the tag server will
		simply display version information and exit.
		\item[\texttt{--logtopics, -T}] - This option requires an argument that is a
		comma delimited string containing the topics that we wish to log.  It only
		affects the default logging service.  If any logging services are configured in
		the configuration file then this will be overridden by the newly created logging
		service.
		\item[\texttt{--verbose, -v}] - This option simply sets the default logging
		topics to ALL.  This has the effect of passing all the messages to the default
		logging service.
		\item[\texttt{--socketname, -S}] - This option requires an argument that
		represents the filename that will be used for the local domain socket.  By
		default it's \texttt{/tmp/opendax}.
		\item[\texttt{--serverip, -I}] - The IP address that the server will listen on.
		Default is 0.0.0.0 which has the effect of listening on all interfaces.
		\item[\texttt{--serverport, -P}] - The TCP port that the server will bind to.
		Default is 7777.
		\item[\texttt{--mod-tag-exclude, -X}] - A space delimited list of module names
		that will be excluded from the module tag creation feature.  Useful for modules
		that will start and stop often and the features of the module tag are not needed
	\end{list}

	Some of this configuration is duplicated in the configuration file.  Here is an
	example configuration file.


	\begin{verbatim}
	-- Main configuration file for the OpenDAX tag server

	-- Configuration files that are common for all programs
	dofile(configdir .. "/log.conf")

	-- The name of the local domain socket that will be created for local
	-- client module communications
	socketname = "/tmp/opendax"

	-- The IP address and TCP port that the server will listen on
	serverip = "0.0.0.0"
	serverport = 7777

	-- Module tag exclusion list. This is a space delimited list of module names that
	-- will be excluded from the module tag feature.  This might be useful for modules
	-- that start and stop often (i.e. modules run from shell scripts) or modules that
	-- have no need of the module tag features.
	mod_tag_exclude = "daxc qdax"

	-- The minimum number of communication buffers that will be maintained
	-- in the system.  The server uses pre-allocated buffers for communication
	-- and this designates the minimum number that will be maintained.
	min_buffers = 5

	\end{verbatim}

	The \texttt{dofile()} functions simply load and run the configuration scripts
	that are given.  This is a very powerful feature of OpenDAX configuration.  It
	allows for common configuration code to be reused in many parts of the OpenDAX
	system.  Here we are just putting all of the message logging configurations in a
	separate file.

    \section{Detailed Description}



	\chapter{Modules}
	Modules are where all the real work gets done on an \opendax{} system.  The server
	is the glue that holds all the modules together but the modules are where it all
	happens.  The \opendax{} distribution comes with a few included modules.  They are
	described in this chapter.  If you have received modules from other sources
	you'll have to consult the documentation that came with that module.

	\section{Common Module Configuration}

	The module configuration file is usually named after the module.  For instance
	the configuration file for the \emph{daxlua} module is called
	\emph{daxlua.conf}.  These files would contain information that is specific to
	each module.  The \emph{daxlua} configuration contains the location of the
	script files to execute as an example.

	There are some configuration options that are common to all modules that are
	included in the distribution.  Some can only be given on the command line,
	others only in the configuration file.

	\begin{list}{}{Command Line Options}
		\item[\texttt{--config, -C}] - The full path to the requested configuration
		file.  This file will be used in lieu of the default configuration.
		\item[\texttt{--confdir, -K}] - If this option is specified the default
		configuration directory will be changed.  The normal default configuration
		filename will be used (i.e. <modulename>.conf) but it will be loaded from this
		directory.
		\item[\texttt{--logtopics, -T}] - This option requires an argument that is a
		comma delimited string containing the topics that we wish to log.  It only
		affects the default logging service.  If any logging services are configured in
		the configuration file then this will be overridden by the newly created logging
		service.
		\item[\texttt{--verbose, -v}] - This option simply sets the default logging
		topics to ALL.  This has the effect of passing all the messages to the default
		logging service.
		\item[\texttt{--socketname, -U}] - This option requires an argument that
		represents the filename that will be used for the local domain socket.  By
		default it's \texttt{/tmp/opendax}.
		\item[\texttt{--serverip, -I}] - The IP address that the module will use for
		network connections.  Default it \texttt{127.0.0.1}.
		\item[\texttt{--serverport, -P}] - The TCP port that the module will use for
		network connections.  Default is 7777.
		\item[\texttt{--server, -S}] - Should be either \texttt{LOCAL} or
		\texttt{NETWORK}.  If set to \texttt{NETWORK} a TCP connection will be made to
		the tag server.  Otherwise the unix domain socket will be used.  Default is
		\texttt{LOCAL}
		\item[--name, -N] - Module name.  The name of the module is used for
		determining the default configuration file, it's sent to the server to identify
		the module to the server and it's passed to the modules configuration file as
		the \texttt{calling\_module} global variable.  It's also used in the message
		logging system for some services.
		\item[--cachesize, -Z] - The \texttt{libdax} library maintains a cache of
		recently used tag definitions so that it does not have to constantly query the
		server for this information every time tags need to be accessed.  The higher
		this number is the less the module will have to query the server but the more
		memory will be used.
		\item[--msgtimeout, -O] - The time before a timeout error will be generated for
		communication with the tag server.  Other timeouts may need to be set for
		specific client module communications, but this one is specifically for the
		client module / tag server communications.
	\end{list}

	The following is an example of a module configuration file.

	\begin{verbatim}
	-- Example module configuration file, showing the configuration
	-- options that are common to all the modules supplied with OpenDAX

	-- Configuration files that are common for all programs
	dofile(configdir .. "/log.conf")

	-- The name can be set here and it will effect the name given to the
	-- server once connected as well as any message logging services that
	-- use the module name.  Obviously the configuration file and the
	-- global variable 'calling_module' will be either the default name of
	-- the module or a name that was passed on the command line.
	name = "monkey"

	-- The name of the local domain socket that will be used to
	-- communicate with a local server
	socketname = "/tmp/opendax"

	-- The IP address and TCP port of the server
	serverip = "0.0.0.0"
	serverport = 7777

	-- When set to NETWORK the client module will use TCP/IP as the
	-- communication mechanism to the server.
	server = "LOCAL"
	--server = "NETWORK"

	-- The size of the tag cache
	cachesize = 32

	-- The server message timeout in mSec
	msgtimeout = 500
	\end{verbatim}

	With the exception of the message logging configuration the command line options
	will override the configuration file options.

	\chapter{Command Line Client Module}
	The command line client module is called 'daxc'.  It is useful for debugging and
	examining or manipulating \opendax{} data in real time.  It can also be used for
	automating some \opendax{} tasks from a shell script.  It can be run interactively,
	it can read input from STDIN or a filename can be passed to it that has lists of
	\texttt{daxc} commands to execute.

	\section{Configuration}
	There is very little configuration that needs to be done to the \texttt{daxc}
	module.  The general configuration that is needed for all modules such as server
	address, ports etc are all that is really required.  There are a few command
	line options that are specific to the \texttt{daxc} module.

	There is no configuration file for \texttt{daxc}.  All of the configuration is
	passed on the command line.

	\begin{verbatim}
	Usage:
	$ daxc [-xiqUIPSZOCKTv] [filename]
	\end{verbatim}

	If \texttt{filename} is given the commands in the file will be executed as
	though
	they were received through \texttt{stdin}.  \texttt{daxc} will also act
	correctly if a \textit{shebang} is used for the file.  So the file can also be
	marked as executable and \texttt{daxc} will act as expected.  Blank lines and
	lines that start with '\#' are ignored.

	\begin{list}{}{daxc Command Line Options}
		\item[--execute, -x] - The execute options should be followed by a command
		that is to be executed by \texttt{daxc}.
		\item[--interactive, -i] - When the module is started with the -f or -x
		options the default behavior of the module is to exit when finished with those
		commands.  If this option is used the program will enter interactive mode after
		those commands are issued.
		\item[--quiet, -q] - Suppresses some output of the program while it is
		executing a file or a command line option.  It does nothing for interactive
		mode.  This is useful for running the module from shell scripts.
	\end{list}

	\section{Commands}

	\subsection{list}
	The \textit{list} command lists information about tags and datatypes.  If the
	command is given no arguments it will simply list all of the tags in the system.
	If followed by the subcommand \textit{type} it will list all of the compound
	data types that are configured in the currently running system.  If
	\textit{type} is followed by the name of one of those compound data types it
	will list the members of the cdt and their data type.

	\subsection{cdt}
	\begin{verbatim}
	Usage:
	daxc>cdt name m1name m1type m1count [m2name] [m2type] [m2count] [...]
	\end{verbatim}
	The \textit{cdt} command is used to create a compound data type.  The command
	should be followed by the name of the new data type and then as many member
	triplicates as are needed to define the type.  The triplicates include the name
	of the cdt member the datatype of that member and the count for the member.  For
	example...

	\begin{verbatim}
	daxc>cdt newType myInt INT 1 myDint DINT 10 myBool BOOL 32
	\end{verbatim}

	This would create a new compound datatype with three members.  The first member
	a single INT named myInt the second an array of 10 DINT's named myDint and the
	last one is an array of 32 BOOLs named myBool.

	You can put as many triplicates after the name as needed to define the cdt.  The
	cdt cannot be redefined, once it is created that's it.  It'll be possible to
	delete unused types at some point but that feature is not yet implemented.

	\subsection{add}
	\begin{verbatim}
	Usage:
	daxc>add tagname type count
	\end{verbatim}

	Adds a tag to the system.  The command should be followed by the name of the new
	tag, the data type of the new tag and the count for the size of the tag.

	\subsection{read}
	\begin{verbatim}
	Usage:
	daxc>read tagname [count]
	\end{verbatim}

	The read command should be followed by the tag that is to be read.  The
	count argument can be given if only a subset of the tags are to be read.

	\subsection{write}
	\begin{verbatim}
	Usage:
	daxc>write tagname value 1 [value2] [...]
	\end{verbatim}

	\subsection{db}
	\begin{verbatim}
	Usage:
	daxc>db tagname [count]
	\end{verbatim}

	\subsection{help}
	\begin{verbatim}
	Usage:
	daxc>help [command]
	\end{verbatim}

	The \textit{help} command followed by the name of another command will list
	information about using that command. Otherwise it simply prints a list of
	commands that help is available on.

	\subsection{exit}
	\begin{verbatim}
	Usage:
	daxc>exit
	\end{verbatim}
	The \textit{exit} command simply exits the daxc module.

	\chapter{Modbus Communications Module}
	The Modbus Client module is a very powerful \modbus{} Protocol communication module.
It is capable of Modbus TCP Client and Server as well as Modbus RTU Master and
Slave.

The basic functional unit of the Modbus module is the \textit{port}.  A port is
associated with a single communication mechanism.  Either a TCP socket or a serial
port.  The port can be configured as either a Master/Client or a Server/Slave.
Multiple ports can be configured so that many serial ports can be used or Multiple
servers can be each assigned a different IP address or port number.  There is no
arbitrary limit on the number of ports that can be configured.

When a port is configured as a master or a client, a list of commands can be
configured that will be sent periodically to access the data in the server or
slave node.  A tag can be created as well that allows other modules in the
system to manually send a modbus request on the port.

When the port is configured as a server or slave, tags are set up that represent
the different register tables (holding, inputs, etc).  These tags would either be arrays
of bits in the case of inputs and coils or 16 bit integers in the case of holding
registers or analog inputs.  Multiple sets of these register tags can be configured
so that multiple nodes or units can be represented with different sets of registers.
If a register is not defined for a specific unit
or node then requests to access those registers will cause unknown function code
errors to be returned to the master/client.

There are also hooks in the server/slave ports that call Lua functions at certain
points in the communication.  This allows the user to intercept a message and
return errors or modify the data on the fly or write tags to synchronize
logic with modbus requests.


\section{Configuration}



	\chapter{PLCTag Communication Module}
	The PLCTag Client module utilizes the \textit{libplctag} library for reading
and writing tags to various PLCs.

PLC tags are defined in the plctag.conf configuration file and are associated
with \opendax{} tags.  The PLCs are polled for their respective tags at intervals
that are configured in the configuration file and updated in the \opendax{}
tag server.

\section{Configuration}

\begin{verbatim}
-- Tag definitions are a table that will be passed to
-- the add_plctag() function later.  The name is
-- arbitrary.   We use tag here but you can use any
-- valid Lua identifier
tag = {}

-- daxtag and count define the tag in the OpenDAX tag server
tag.daxtag = "TestUINT" -- Name of the tag in OpenDAX.
tag.count = 5  -- Number of elements in the above tag

-- elem_size overrides the size of the element in the PLC.
tag.elem_size = 4

-- Used for rearranging bytes when access to PLC tags is raw
tag.raw_byte_map = {1, 0, 3, 2}

-- plctag is the attribute definition of the PLC tag that defines the
-- tag in the libplctag library.  See the libplctag documentation for
-- a complete description. 

-- tag.plctag = "protocol=ab-eip&gateway=172.16.4.1&path=1,1&
--               cpu=LGX&elem_count=10&name=TestDINT"
tag.plctag = "protocol=modbus_tcp&gateway=127.0.0.1&path=1&
              elem_size=4&elem_count=10&name=hr0"


tag.read_update = 1000  -- amount of time between reads from the PLC
tag.write_update = 500  -- amount of time between writes to the PLC

-- Finally we create the tag in the module configuration
add_plctag(tag)
\end{verbatim}

The main configuration of the PLCTag module is based around the 
definition of 'tags.'  This can be a little confusing since we
use the name \textit{tag} as the definition of a single unit of
data within \opendax{}.  Here we differentiate between a \textit{PLC tag} and
an \opendax{} \textit{tag}.

Tags are defined in the configuration with a table that uses
strings as the index.  This table is set up and then passed to
\texttt{add\_tag()} configuration function.  The table can then
be reused or another table can be defined.

A PLC tag is defined with a URL like string that defines the tag
for the libplctag library.  \opendax{} simply passes this string to
the libplctag library.  See the libplctag documentation at...

\texttt{https://github.com/libplctag/libplctag/wiki/Tag-String-Attributes}

This string is configured by setting the \texttt{plctag}
member of the tag table in the Lua configuration file.

The \opendax{} tag is defined with the \texttt{daxtag} member of the
tag table.  The PLCTag module does not create this tag.  It can
be created in an \texttt{init\_hook()} function or in another module.
The tag need not exist when the PLCTag module first starts but errors
will be generated in the log each time the module tries to write the
tag to the tagserver, until the tag is created.

The \texttt{count} member defines the number of items in the \opendax
tag we will use.  This count is the number of items that we will try
to read from the PLC Tag, regardless of how many elements the PLCTag
represents.

The \texttt{elem\_size} member defines the assumed element size of
the PLC tag.  If this is not present or set to zero then the actual
element size will be determined from the libplctag library.  This 
can be overridden to make it possible for the tag in \opendax{} to be
larger than the tag in the PLC.  One use for this would be in the case
where a 32 bit floating point number is being represented with two
consecutive 16 bit numbers in the PLC.  In this case the element size
can be overridden to 4 and this will work as expected.

The \texttt{raw\_byte\_map} member is an optional definition
of the requested byte ordering when the conversion is done raw.
\texttt{elem\_size} must be defined for this to work and the byte mapping
array will be set equal to the element size and either truncated if too
large or filled in with values equal to the index.  The value being 
the same as the index has the same effect as no byte mapping at all.
Each number in the byte map array represents the byte in the PLC buffer
that will be written into the byte of the \opendax{} tag buffer for
that place in the array.  For example if the data that is read from the plc
is 11, 22, 33, 44 and the byte map array is {1,0,3,2} then the data that
will be written to the \opendax{} tag buffer will be 22, 11, 44, 33.

The \texttt{read\_update} member defines the read polling time
for the tag in milliseconds.  The libplctag library will poll this
tag at that rate and when the tag is read it will be written to the
\opendax{} tag server.

The \texttt{write\_update} member defines the rate in which the writes
to the PLC will happen.  When defined tags change in the tag server an
event is generated in the PLCTag module.  The write is queued in the
libplctag library and will be written to the PLC at the expiration of
the write\_update time.  This allows multiple values to be written to
the PLC at once.  This can make the communication to the PLC much more
efficient since many PLC allow grouping of data in messages, like writing
multiple registers in \modbus{}.

\section{Detailed Description}

Tags are mapped according to a couple of criteria.  The tag data is
read into a buffer within libplctag and then are translated into the 
proper data type for \opendax{}.  The conversion is based on the data
type of the \opendax{} tag.  If the \opendax{} tag is a INT 
(16 bit signed integer) and the PLC tag is of type DINT (32 bit
signed integer) the conversion will work as expected as long as the
value in the PLC is less than or equal to 65536.  At that point the
the value will roll over.  This is the case for single tags.  If the
tags are arrays then it gets more confusing.

The array of data is still read into the libplctag library as a whole
and is represented as a buffer of bytes.  The conversion will still
take place based on the size of the \opendax{} tag but the offset into
the PLC tag buffer will be based on the element size.  This element
size is determined by the PLC tag definition in the library and may be
dependent on the type of PLC as well as the type of tag.  If the number
of elements in the tag and the \textit{count} of the \opendax{} tag are 
the same and the data types are the same then there is no issue.

If the size of the \opendax{} tag is greater than the element size of the
PLC tag then the element size can be overridden in the configuration.
This would allow us to have 8 32 bit floating point numbers represented
in the PLC as 16 16 bit integers or 32 8 bit bytes.  The element count
in the \texttt{plctag} configuration would need to be set appropriately
for this to work correctly.

As an example, let's say that we want to read 8 REAL (32 bit floats) into
\opendax{} from a \modbus{} PLC.  The floats are stored in the modbus PLC
as 16 consecutive registers in the holding register block.  We would
configure the \texttt{plctag} member of the configuration to read these
16 registers and we'd configure the \texttt{daxtag} and \texttt{count}
members as the name of a REAL tag in \opendax{} with a count of 8.  Then
since the normal element size would normally be 2 for the PLC we'll have
override this with a value of 4.  This would work as we expect.

The conversion is always done with the type found in the \opendax{} tag, and
is carried out based on the count found in the \opendax{} tag.  If the count
is left off it will be set to 1.

If the \opendax{} tag is a \textit{Custom Data Type} (CDT) then no data type
conversions will be done and the data will be simply copied byte by byte.  If
the \opendax{} tag is a CDT and the PLC tag is a ControlLogix UDT and they
are structured correctly then it should work.  Testing should be done since
the way PLCs pack structured data and the way \opendax{} packs structured
data may not be the same.  ControlLogix PLCs pack consecutive booleans into
a 32 bit space whereas \opendax{} packs them in 8 bit spaces.  This might mean
that there needs to be 'filler' bytes added to align the rest of the data
type.  The user will have to have a good understanding of the way these
data types are structured on both ends for this to work.

Of course, it is entirely possible to pack data into arrays of bytes in
the PLC that are represented by CDTs in \opendax{} and the reverse is
also true.  Byte ordering is an issue here and we are working on a way
to deal with this problem.  Byte ordering for other types works fine
within libplctag but the raw byte copy that happens for CDTs does not
deal with byte ordering properly.  See the raw byte mapping feature
for a way to fix byte ordering issues in CDT conversions.

The offset into the PLC tag buffer is based on the element size and the 
\opendax{} tag count.  Each iteration through the \opendax{} tag
the offset is incremented by the element size.  This is either the size determined
by the PLC tag when it is created or from the \texttt{elem\_size} configuration
if it is given.  It is important that the PLC tag buffer be the same size
or larger than the \opendax{} tag count x the element size.  If not then
overruns will happen.  These overruns are safe since the libplctag library
checks for this condition and doesn't allow unsafe memory accesses but
how this data is returned is not clear for all data types.  Typically
a minimum value is returned but this could change.  Check the \opendax{}
log for these errors if the data you are receiving does not seem correct.

Boolean values work as expected.  If the \opendax{} tag is a BOOL then
the count given by the tag will represent bits.  This will behave as expected
if the PLC tag is a single 32 bit unsigned integer and the \opendax{} tag is
an array of 32 bits.  Each \opendax{} BOOL will represent each bit position
within the PLC integer.

\section{Notes}

When reading structured data from AB PLCs is will usually be helpful to
put padding bits and bytes in the CDT that you use in \opendax{}.  If it's
just a matter or rearranging the bytes then the byte ordering feature can
be used but sometimes it's just easier to pad the bytes and not worry about
it.

For example, to read a timer from a ControLogix PLC the following CDT can 
be used.

\begin{verbatim}
cdt_create("lgx_timer", {{"pad", "BYTE", 3},
                         {"pad2", "BOOL", 5},
                         {"DN", "BOOL", 1},
                         {"TT", "BOOL", 1},
                         {"EN", "BOOL", 1},
                         {"PRE", "DINT", 1},
                         {"ACC", "DINT", 1}})
\end{verbatim}

You can see that we padded the beginning of the CDT with three bytes because
the AB PLC uses these internally and almost everything in ControlLogix is
aligned on 32 bit boundaries.  Then we pad with 5 bits because the three
bits that we are interested in are the high order bits in that fourth byte.

It takes some experimentation to get these correct.  A good place ot start
is to watch the daxplctag module logs and pay attention to the line where
the PLC tag is created.  The size of the PLC element will be displayed.  Create
the \opendax{} CDT with that many bytes and do some testing.  Then adjust the
CDT until the data is read properly.

If it becomes apparent that the byte ordering is wrong (usually the case with
\modbus{}) then byte ordering can be used to fix it all.

	\chapter{Lua Scripting Module}
	The daxlua module is logic module that uses the Lua programming language to
write scripts that are executed and have access to the \opendax tags.

Scripts are written in normal text files and stored in any arbitrary location in
the filesystem.  They can be run either on a periodic basis or they can be triggered
by tag events.

To run scripts on a periodic basis they need to be assigned to an \textit{interval thread}.
Interval threads sleep for a period of time and then wake up and run all of the scripts
that are assigned to them.  The user has complete control over all of the interval threads
that are created in the system and which scripts are assigned to them.

The other method of running scripts is to assign a trigger to them.  Triggers in the \textit{daxlua}
module are nothing more than \opendax events.  When the event is received from the server,
the associated script will be executed.

These triggered scripts are not executed directly but rather, they are stored on an execution
queue and run by one of several \textit{queue threads}.  These threads are started early in the system
and they simply wait until a script is placed on the queue, then one of the threads will retrieve that
script from the queue and execute it.  The number of queues that are started by the module is a
configuration for the user.

Scripts can also be enabled or disabled in one of several ways.  The enabled state of the script can
be chosen in the main configuration file and that will be the state of the script when started.  Scripts
can be enabled or disabled via triggers that are nothing more than \opendax events from the tag server, or
they can be enabled / disabled from one another.  There is also a Lua function that allows scripts
to disable themselves.  This is handy when a script simply needs to only be scheduled for a short time, until
the job is done and then disable themselves.

\section{Configuration}

The module looks for the file \texttt{daxlua.conf}.  Not having a proper configuration is a fatal error
as it makes no sense to run with defaults.  The normal module configuration options apply to daxlua
as well as many other.

\begin{verbatim}
	-- Number of worker threads for handling event trigger scripts
	event_thread_count = 8
	-- Size of the event queue.  This is the queue that holds the
    -- scripts between the time the server sends the event and
    -- the worker threads can execute the script.
	event_queue_size = 128

	-- Adds an interval thread.  These threads will run all of
	-- the scripts that are assigned to them by name every time
	-- the given time has elapsed

	add_interval_thread("fast", 500)
	add_interval_thread("1sec", 1000)
	add_interval_thread("slow", 2000)

\end{verbatim}

The \texttt{event\_thread\_count} attribute is the number of threads that will be created to handle
scripts that are going to be run by triggers.  Scripts are not released from the thread that
they are assigned to until they have completed so if there are many scripts that take a long
time to run then it would make sense to have more scripts.  For a few very short duration scripts
fewer threads can be used.  Some experimentation may be necessary to determine the optimal thread
count for the users application.

The \texttt{event\_queue\_size} attribute is the size of the queue that will be used to hold events that
need to be processed by the event threads.  If this queue fills up then events will be lost and
errors will be placed in the log file.  This is nothing more than an array of pointers so it is
fairly inexpensive in terms of memory usage.  The queue should be big enough to handle all the events
that might happen simultaneously.

The \texttt{add\_interval\_thread()} configuration function causes the module to create a thread
that can later have scripts assigned to them.  There are two arguments to the function.  The first
is the name of the thread (this will be referred to later by the scripts) and the desired interval.

The interval threads are at the mercy of the operating system in terms of their actual scheduling.
If realtime deterministic scheduling is required then this is probably not the mechanism.  There
are ways within the threads to determine the time between executions of the thread so that fairly good
integrations can be done with scripts but it may not be deterministic enough for every use case.

\begin{verbatim}
s = {}

s.name = "demo1"    -- Unique name for the script
s.filename = root.."demo1.lua"  -- The path to the script
s.enable = false    -- If false the script will not run
s.fail_on_error = true
s.thread = "fast"   -- The name of the interval thread
                    -- that we wish to be assigned to
s.enable_trigger = {tag="s_enable", count=0, type="SET", value=0}
s.disable_trigger = {tag="s_disable", count=0, type="SET", value=0}

-- Once the table has been configured, pass it to 'add_script()'
add_script(s)

\end{verbatim}

Above is an example of how scripts are configured.  First we create a table...

\begin{verbatim}
	s = {}
\end{verbatim}

The we fill out that table with entries.  Most of these entries should be self-explanatory.  \texttt{name}
is a unique name given to the script.  This is necessary for referring to this particular script
within the configuration but also from within the scripts.

The filename should be a proper filesystem location of the Lua file that we wish to associate with this script.
The \texttt{root} variable that is shown is an easy way to define a path to where all of our scripts are
located.  Since our configuration is nothing more than a Lua script itself, we can use concatenation
tricks like this to save us some work.  We can also use loops and other language features if it is
appropriate.

If \texttt{enable} is set to \texttt{false} then the script will be assigned to the proper script but
it will not be executed until it is enabled my some mechanism.  Either by another script or a trigger.

If \texttt{fail\_on\_error} is set to true the script will fail (no longer be executed) if
an error is raised by the script.  Errors that are caught in the scipr (by pcall() or similar)
will not cause the failure.  To allow a script to continue to run after an error is caught
this will need to be set to false.  The default is true.

\texttt{thread} is the name of the interval thread, that we wish to assign the script to.  Each
script can only be assigned to one thread.

The \texttt{enable\_trigger} and \texttt{disable\_trigger} attributes are tables that represent the \opendax
event that we wish to enable or disable the script.  Four pieces of information define an event but
not all are appropriate for each event type.  Detailed descriptions of the event system are given
elsewhere in this manual\footnote{Probably not done yet, documentation is still a work in progress}.

To configure a triggered script the \textit{trigger} attribute should be set.  This attribute is
similar to the enable / disable triggers described above except that when the event is received
from the tag server the script is placed on the queue to be executed.

\begin{verbatim}
	s.trigger = {tag="triggerdemo", count=0, type="CHANGE", value=0}
    s.auto_run = false
\end{verbatim}

The \textit{trigger} attribute and the \textit{thread} attribute can be set for each script and this
is not an error at the moment.  This would essentially cause the script to be executed within
the interval thread to which it is assigned as well as be executed when triggered.  A warning will
be generated in the log since this is probably not the desired behavior but it may work.  It should
be noted that the script only exists in one place so the context between executions will be the
same regardless of whether it was executed by the interval thread or by a trigger.  It has not
been tested so you are on your own here.

Any data that is associated with the trigger event will be stored in the global variable
\texttt{\_trigger\_data} so the Lua script will have access to that data without having to query
the tagserver.

If \texttt{auto\_run} is set to true the triggered script will be run at least once during
module startup.  This initial run will not be the result of the trigger event being received
from the server so the \texttt{\_trigger\_data} variable will be \textit{nil}.  The default
is false.

Once we have all the data in the script table we can pass it to the \texttt{add\_script()} configuration
function to actually create the script.

There two types of global data that can be assigned to the scripts before and/or after they are
executed.  The first kind of global data is a \textit{global tag}.  These are \opendax tags that will
either be read from the tagserver just before the script is executed, written to the tagserver
after the execution of the script or both.

To add a global tag to the system, use the folowing function...

\begin{verbatim}
	add_global_tag(s.name, "DEMOTAG", "DAXTAG", READ + WRITE)
\end{verbatim}

The first argument to the function is the name of the script that we wish to assign the tag to.
Since we still have the name stored in the table we simply use that value here.  Any string can
be used as long as it matches a script that has already been created with \texttt{add\_script()}.

The second argument to the \texttt{add\_global\_tag()} function is the name that will be given to
the global variable in Lua.

The third argument is the \opendax tag that we wish associate this data with.  It can be any
valid string that represents a tag in \opendax.  Arrays and CDTs are allowed and the module
will convert them to the appropriate Lua value / object.  For example, say the tag \texttt{DEMOTAG}
exists in the tagserver and it is an array of 8 double integers (DINT).  If we use "DEMOTAG" as the
name here then we will get a table in Lua with all 8 of the integers.  If we use "DEMOTAG[4]" here
we will get a single integer that is the fifth element in the array.\footnote{It is worth remembering
Lua starts numbering array elements at 1 instead of 0.  This is different than how it is represented in
the tagserver.  DEMOTAG[4] would be the fifth element in the tagserver but the fourth in Lua.}

The fourth argument is the mode of the global.  It is either READ, WRITE or the sum of those.  If
it is set to READ then the tagserver will be queried for the given tag just before the script
is executed and the value placed in the global Lua variable.  If WRITE is given then the value
will be read from the global Lua variable and then written out to the tagserver immediately after
the script executes.  It should be noted that the value only changes within the Lua script as
the script is executed.  It does not change in the tag server until after the script finishes.
There is a mechanism within the scripts that we can use to write values to the tagserver during
the script execution.  If the mode is set to READ + WRITE then it will be read from the server
just before execution and then written back to the server when the script is done.

In the given case if we write a value to the variable \texttt{DEMOTAG} in our Lua script
that value will be written to the \opendax tag, \texttt{DAXTAG}.

The same Lua file can be assigned to different scripts and different tags can be read from the
tag server into the same global Lua variable.  This allows us to write the script once but use
it for multiple different tags within the tagserver.

As many global tags can be assigned to a script as we wish but keep in mind it causes a read and/or
write cycle to the server so performance could become an issue.\footnote{Some optimizations can be done
here with tag groups but this has not yet been implemented.  The hope is that it will be invisible
to the user once it is done.}

Another piece of data that can be assigned to the individual scripts is a \textit{global static}
variable.  This is simply a value that is configured for the script that will always be
available when the script runs.

\begin{verbatim}
	add_global_static(s.name, "VARNAME1", true)
	add_global_static(s.name, "VARNAME2", 123.2)
	add_global_static(s.name, "VARNAME3", "some string")
\end{verbatim}

The first argument to \texttt{add\_global\_static()} is the name of the script.  The second
argument is the name that will be given to the global variable within the Lua script and
the third is the value.  This is a way to pass arguments to a script.  It could be used
to differentiate between scripts.  It's more efficient than global tags since there is no
interaction with the server but it is static.  The value passed can be a boolean, a number,
a string or \textit{nil} and that is all.  It turns out that it is surprisingly difficult
to move complex values like Lua tables from one Lua context (the configuration) to another
(the script).  If this becomes necessary in the future it can be added but it seems like it's
not worth the effort at this point.

Global static variables can be changed within the script as its executing as can any Lua
global variable but the next time the script is executed it will be returned to the
configured value.

Using Lua as our configuration language makes this a very powerful system.  Any Lua language
feature can be used to generate the tables and call the functions that configure the module.
Including loops, flow control, functions and file inclusions.
This is true of all the modules in \opendax and it is the 'killer feature' of
the system.  It can be confusing but such is the case for many things that are this powerful
and flexible.

\section{Script Writing}

The Lua scripts that we execute in this module should be scripts that execute and then
end.  If a script stays running for a long time it could hang up the thread to which it
is assigned and block other scripts from running.

The exectution order of the scripts is same as the order in which they are added in the
configuration file by the \texttt{add\_script()} function.

Execution of the script is the same whether it is run from an interval thread or if
it is triggered by an event.  The first thing that happens is the time is
recorded
that the script is being executed.  Then all of the global
tags and data are put into the script.  This may include tags that are read from the
system as well as some variables that represent status and the static globals.
Then the script is executed and once the script finishes the global tag variables
that are configured are written out to the tagserver and some housekeeping is done such
incrementing counters and storing timing information.

There are a handful of global variables and functions that are automatically available
to the Lua script when it runs.  Some were already discussed in the configuration
section.  The rest will be described shortly.

Triggered scripts will be run once, automatically by the system when they
are created.  This is to give the scripts a chance to do any initialization
that may necessary.  If the \texttt{\_first\_run} variable is \textit{true} for
these scripts then you can know that it was not called because an event
was received, rather because it was automatically run when created.  The
reason this was done is so that these scripts can have a chance to create
the actual tags that they are being triggered from or do any other
initialization that needs to be done before the system gets
started.\footnote{We may make this configurable in the future}

\subsection{\opendax Interface Functions}

There is a Lua package included with \opendax that can be used to write entire \opendax
modules in Lua.
In that package is a set of functions and constants that allow the programmer to access
and manipulate data in the tagserver.  The library that comprises that package is also
used in this module to give the Lua script access to the \opendax data.  Not all of
the functions are available however.  For example, it does not make sense to create
a connection to the server since that has already been done.  Waiting on events also
does not make sense because we don't want our scripts to run very long.

The subset of functions that are available to Lua scripts in this module are...

\begin{itemize}
	\item \textbf{cdt\_create(typename, members)} - The function is used to
	create a
	\textit{Compound Datatype}.  The first argument should be a string that
	will be used as the name of the CDT.  The second argument is a table of
	tables that defines the members of the CDT.  An example of a member
    table is given below...
    \begin{verbatim}
        members = {{"Name", "DataType", count},
                   {"AnotherNmae", "DataType", count}}
    \end{verbatim}

    This function raises errors on failure and returns a single integer that
    represents the datatype and can be used to create tags.

	\item \textbf{tag\_add(name, type, <count>)} - Adds a tag to the tagserver
	database.  The first argument is a string that
    represents the name of the new tag.  The second argument can either be an
    integer or a string that represents the data type of the tag.  The third
    argument represents the number of items created for the datatype.
    If this number is greater than 1 then an array is created.  If the count is
    not given then 1 is assumed.

    This function returns nothing on success and raises errors otherwise.

	\item \textbf{tag\_get(tag)} - Retrieve the definition of the given tag.
	The function takes a single
    argument that can either be the tagname as a string or the tag index as an
    integer.

    The function returns three values that represent the tag, name, type and
    count.

	\item \textbf{tag\_handle(tag, <count>)} - Retrieve the definition of the
	given tag.  The function takes at least one
    argument that can either be the tagname as a string or the tag index as an
    integer.  The optional second argument is the number of items we wish
    to get.

    The value that is returned is userdata that means nothing to the Lua script
    but that can be passed to the \texttt{tag\_read} and \texttt{tag\_write}
    functions.  Using handles for reading and writing tags is much more
    efficient
    than having to find the tag based on it's name every time.

	\item \textbf{tag\_read(tag, <count>)} - Read and return the value(s) of
	the given tag.  The first argument is a
    string representing the tag that we wish to read, or a handle that was
    received from  the \texttt{tag\_handle} function.  The second, optional,
    argument is the number of members that we want to read.  If the first
    argument is a handle then the count will be ignored since the handle
    fully defines the data that we want to read

    The return value depends the type and size of the tag.

	\item \textbf{tag\_write(tag, val)} - Write the value to the given tag.
	The first argument is a string
    representing the tag, or a handle that was received from  the
    \texttt{tag\_handle} function.

    The function returns nothing and raises errors on failure.

	\item \textbf{log(topic, message)} - Logs the given message to the topic.
	A list of logging topics is given below.  These are consistent throughout
	the \opendax system.

    \begin{verbatim}
        LOG_MINOR
        LOG_MAJOR
        LOG_WARN
        LOG_ERROR
        LOG_FATAL
        LOG_MODULE
        LOG_COMM
        LOG_MSG
        LOG_MSGERR
        LOG_CONFIG
        LOG_PROTOCOL
        LOG_INFO
        LOG_DEBUG
        LOG_LOGIC
        LOG_LOGICERR
        LOG_USER1
        LOG_USER2
        LOG_USER3
        LOG_USER4
        LOG_USER5
        LOG_USER6
        LOG_USER7
        LOG_USER8
        LOG_ALL
    \end{verbatim}

\end{itemize}


\subsection{daxlua Specific Functions}

Other functions that are added to the context of a script here are...

\begin{itemize}
	\item \textbf{disable\_self()} - Causes the currently executing script to be disabled
	\item \textbf{get\_executions()} - Returns the number of times our script has been executed
	\item \textbf{get\_name()} - Returns the currently executing scripts name as given in
	the configuration file.
	\item \textbf{get\_filename()} - Returns the file name of the script
	\item \textbf{get\_lastscan()} - Returns the time in microseconds that the script ran last time.
	This is roughly the number of microseconds that the system has been running.
	\item \textbf{get\_thisscan()} - Returns the time in microseconds that the script execution
	was started. This is roughly the number of microseconds that the system has been running.
	\item \textbf{get\_interval()} - Returns the time in microseconds since last time we ran.
	This is the difference of the above two times. Note that this is not how long it took the script
	to run.  It's the period between the two.  It should be fairly close to the configured interval
	that was configured for the thread.  It might be meaningless for triggered scripts but it is
	there just in case.
	\item \textbf{get\_script\_id(name)} - Returns the integer id of the
	script given by
	name or nil if not found.
	\item \textbf{get\_script\_name(id)} - Returns the name of the script
	given by
	the id or nil if out of bounds
	\item \textbf{disable\_script(name or id)} - Disables the given script.
	Returns
	the id of the script that was disabled or nil if it failed.
	\item \textbf{enable\_script(name or id)} - Enables the given script.
	Returns
	the id of the script that was enabled or nil if it failed.
\end{itemize}

Also the standard Lua packages \textit{base}, \textit{table}, \textit{string}
and \textit{math} are included.

\subsection{Global Data}

The module add some global variables to the script each time it is run.

\begin{itemize}
    \item \textbf{\_firstrun} - This variable is set to \textit{true} the
    first time the script is executed.  It will be \textit{false} afterwards.

    \item \textbf{\_trigger\_data} - Contains the data that was sent by the
    tagserver for the event.  For example, if you triggered this
    script with a CHANGE event on a tag named \textit{trigger} that is a single
    INT tag.  If some other module changes \textit{trigger} to the value of
    1234 then when your script is called, \texttt{\_trigger\_data} will be
    equal to 1234.  Arrays and CDTs can also be used as trigger events and they
    should behave as you would expect here as well. This is only available for
    scripts that were triggered and executed from the queue.  Interval scripts
    will return \textit{nill} for this variable.  It will also return
    \textit{nil} on the first run of the script since the first run is not
    triggered.
\end{itemize}

Other global variables can be configured for each script.  These are
\textit{global static} variables and \textit{global tag} variables.  These
are described in the configuration section above.

Global data that is created in the Lua script will survive between executions
since the Lua state is maintained.  This can be used to keep track of state
between executions.

An interval script may wish to run five times each time it is enabled and
then disable itself.  That could be done like this...

\begin{verbatim}
if _firstrun then
   x = 0
end

--Do some stuff here...
x = x+1

if x>=5 then
    x = 0
    disable_self()
end
\end{verbatim}

x is initialized in a block of code that only runs the first time the script
is executed.  Then it is used to keep track of how many times we have been
run.  Once we have run five times we reset x and then disable ourselves.
The next time the script is enabled (either by trigger or another script)
it will run five more times.

If this is not what you want then you will have to be careful to initialize
the global data each time the script is executed.

\subsubsection*{Notes}

Any errors that are allowed to propagate up to the calling module
will be marked as \textit{failed} and will never be executed again. Any
code within the scripts that could cause failures that may not be permanent,
(like trying to read from tags that don't yet exist) will have to be caught
or the script will be permanently disabled.
This behavior can be changed by setting the \texttt{.fail\_on\_error}
entry in the script table to false.  If this set to false the script
will continue to execute and fail.  Entries will be put in the error
log for each occurance in this case.

A future feature addition will be the ability to reload scripts from disk
while the module is running.  Once this feature is added the \textit{failed}
flag will be reset when a new script is loaded.

Be careful with triggered scripts that execute very frequently.  If several
event triggered scripts are overlapping because events are coming in faster
than the scripts can be run then the \texttt{\_trigger\_data} value may
not be reliable.  There are protections to keep the same script from running
at the same time but there are no protections for this data.  If the
script is in the queue multiple times then all of the scripts will run
with the latest \texttt{\_trigger\_data} and scripts that were waiting in
the queue when subsequent events arrived will lose their data.

	\chapter{MQTT Module}
	The MQTT Client module allows the user to connect to an MQTT broker and associate tags with topics.

\opendax tags can be associated with either a publisher or a subscriber and Lua functions are used to format the data from the broker to the \opendax tags.  These formatting functions have the ability to add, read and write tags so the mechanism is very powerful.

\section{Subscriptions}

Subscriptions can subscribe to any topic that is allowed by MQTT.  Wildcards can be used.  The actual topic that is received will be passed to the formatting function that are configured for the subscription so the formatting function can make decisions based on the actual received topic.

The formatting functions have the ability to create, read and write any tag, regardless of whether they are configured along with the subscription.  Tags that are configured in the subscription are more efficient because a buffer is pre-allocated to hold the data and the tag handles are stored.

If no tags are associated with the subscription then it's assumed that the formatting function will write the tag data to the server.  It is not required that the formatting function write any data to the tag server.  It may be that the formatting function decides based on the data to make no changes or the formatting function may do nothing other than write the data to the message logger with the Lua \texttt{print}\footnote{The print function is overridden by the module and will pass the given string to the message logger with the topic of ERR\_LOGIC} function.

When a formatting function and tags are assigned, the module expects the formatting function to return values that will then be converted and assigned to the tags.  It is important that the function return the proper type and number of values that match the configured tags.  Warnings will be written to the message logger if the number of tags configured does not match the number of values returned.

If more than one tag is associated with the subscription then a tag data group is used for the communication to the tag server.  Otherwise, a simple \texttt{tag\_write} is used.  Any tag can be associated with the subscription including arrays and tags that have compound data types.  If subsets of arrays or parts of tags are to be written then a Lua formatting function that directly writes the tags will have to be used.

The simplest (and fastest) formatting method happens when no formatting function is specified.  If no formatter is specified then a simple raw copy of the MQTT message payload is copied to the tag.  Byte ordering issues are ignored and no attempt is made to check the data.  This is extremely fast but obviously not very flexible.  Mirroring a set of tags between two \opendax systems would be very efficient if done this way.

\section{Publishing}

Publishing is similar except that the formatting function should return a single string that will be the data to be published.  If tags are given they will be read and passed to the formatting function.  If no formatting function is given then a raw copy of the tag data will be published to the topic.  At least one tag must be configured to use as a trigger.  The trigger is what causes the publication to happen.  A specific trigger tag can be given.  This will be used to setup a regular dax event and hits on that event will cause the publication.  The type of trigger can also be set and if necessary, a value for those event types that require a value (i.e. Greater Than).  The trigger tag need not be one of the tags that will be part of the publication.  If a trigger tag is not explicitly configured then the first tag that is assigned to the publisher will be the trigger.  If not type is given then the WRITE event type will be used.  If no trigger tag is given and no tags are assigned, an error is raised and the publisher will not be enabled.

The formatting function accepts two parameters, the topic name and the tag values.  If there is a single tag configured for the publisher then the value of that tag will be passed to the function.  If there are more than one tag then an array of values corresponding to the tags will be passed as the value.  If not tags are configured nil will be passed.  It is expected that the formatting function will return a string.  The string can be a binary string that contains zeros.  The \texttt{struct.pack} function may be very useful here.  Also the \texttt{string.format} function.

\end{document}

