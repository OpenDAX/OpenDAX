\documentclass[10pt,letterpaper]{report}

\usepackage[american]{babel}
\usepackage{amsmath}
\usepackage{amsfonts}
\usepackage{amssymb}
\usepackage{setspace}
\usepackage{parskip}
\usepackage[dvips]{graphicx}
\usepackage{float}
\usepackage{makeidx}
\usepackage{verbatim}


\setlength\parskip{10pt}
\setcounter{secnumdepth}{3}

\makeatletter
\def\thickhrulefill{\leavevmode \leaders \hrule height 1pt\hfill \kern \z@}
\renewcommand{\maketitle}{\begin{titlepage}%
    \let\footnotesize\small
    \let\footnoterule\relax
    \parindent \z@
    \reset@font
    \null\vfil
    \begin{flushleft}
      \huge \@title
    \end{flushleft}
    \par
    \hrule height 1pt
    \par
    \begin{flushright}
      \LARGE \@author \par
    \end{flushright}
    \vskip 60\p@
    \vfil\null
  \end{titlepage}%
  \setcounter{footnote}{0}%
}

\makeatother
\makeindex

\def\opendax{\textit{OpenDAX}}
\def\modbus{\textit{Modbus}$\textsuperscript{\textregistered}$}


\title{OpenDAX User's Manual}
\date{June 13, 2008}
\author{Phil Birkelbach}


\begin{document}
	\pagenumbering{roman}
	\maketitle

	\begin{flushleft}
		Copyright \textcopyright 2008 - Phil Birkelbach\linebreak
		All Rights Reserved
	\end{flushleft}

	\tableofcontents
	\newpage
	\pagenumbering{arabic}
	\chapter{Introduction}
	\opendax{} is an open source, modular, data acquisition and control system
	framework. It is licensed under the GPL (GNU Public License) and therefore is
	completely free to use and modify. \opendax{} is written primarily for Linux.
	There should be no reason that it wouldn't compile on other *nix like operating
	systems but for now we are concentrating on Linux.

	\opendax{} could be used for anything from controlling the air conditioner in a
	home to controlling an entire industrial facility. Depending on what modules are
	installed and run it could be used as the firmware for a dedicated Programable
	Logic Controller (PLC) or a Home Automation system. It could loosely be compared
	to DCS (Distributed Control System) or a SCADA (Supervisory Control and Data
	Acquisition) system. Eventually the system would be able to scale up to a
	several hundred thousand tag system. At this moment the code is far to immature
	to be used for anything that required reliability but we hope to get it to that
	point sooner or later. Much will depend on how many developers take up the
	challenge to help work on the code.

	DAX stands for Data Acquisition and eXchange. The system works by combining a
	master program (opendax), a real time database server (tagserver) a library
	(libdax) and set of modules. The opendax server handles the starting, stopping
	and monitoring of process that make up a particular OpenDAX system.  It is
	actually an optional part of the system.  The user can start and stop each
	process manually if desired.

	The tag server is the program that manages the real time data that is used
	throughout the system.  It could hold the temperature or pressure of a process,
	the status of a switch or some command data from the Human Interface to a logic
	module.  Each piece of information is called a 'Tag'  Each tag can be a single
	value or an array of values.  (Multi dimensional arrays are not supported.)
	Each tag is defined as a given data type.  There are several base data types.
	These represent simple number types like INT or FLOAT.  It is possible to create
	compound
	data types that are collections of the base data types and other compound data
	types.

	The modules do all the work and communicate with the \texttt{tagserver} through
	the \texttt{libdax} library. There could be modules for reading and writing to
	I/O points, data logging, alarming, machine interface, and logic. The primary
	interface to the tag server is the \texttt{daxc} module, which is a command line
	interface module that can
	be used to read and write tags, check status and do all other types of
	maintenance in the tag server.

	There is no requirement that all of the modules have to be on the same machine.
	There is also no requirement that the modules be on the machine with the tag
	server.  Any process that is to be started by the \texttt{opendax} server will
	have to be on the same computer.  Modules communicate to the server by one of
	two mechanisms.  Either a TCP/IP connection or a \textit{local domain socket}.
	The \textit{local domain socket} has the advantage of being very fast because
	the operating system kernel simply copies memory from one process to another.
	The disadvantage is that the module that wants to use this communication
	mechanism must be on the same machine as the tag server. If the module is on
	another machine then it will have to use the TCP/IP protocol to communicate with
	the tag server.

	The system is designed so that most of the work happens in the client library
	and the modules.  Since the tag
	server is a bottleneck for the entire system, any work that could be offloaded
	to the client library and the
	modules was.

	\chapter{Getting Started}
	\section{Installation}

OpenDAX uses the CMake build system generator.  You'll need to install CMake
on your system.

You will also need the Lua development libraries installed.  Most 
distributions have versions of Lua that will work.  The currently supported
versions of Lua are 5.3 and greater.
If you compile and install Lua from the source files you will need to add -FPIC 
compiler flag to the build.

\begin{verbatim}
make MYCFLAGS="-fPIC" linux
\end{verbatim}

Once you have CMake and the Lua libraries installed you can download and build
OpenDAX.  First clone the repository...

\begin{verbatim}
git clone https://github.com/OpenDAX/OpenDAX.git
\end{verbatim}

This should create the OpenDAX directory.  Now do the following...
\begin{verbatim}
mkdir build
cd build
cmake ..
make
make test
\end{verbatim}

If all the tests pass you can install with \ldots

\begin{verbatim}
sudo make install
sudo ldconfig
\end{verbatim}


	\section{Configuration}
	There are two ways to pass configuration information into any given \opendax
	program.  The first is simply to pass command line arguments and the second is
	via a configuration file.  There is likely some overlap between what options can
	be passed with either method but some options only make sense for the command
	line (an example is the \texttt{config} option that specifies which
	configuration file to use).  Some options might be too complex to put on the
	command line and will require a configuration file.

	If a configuration attribute is present on the command line it will take
	precedence over that same attribute given in the configuration files.  An
	exception to this is the logging configuration.  Once logging services are added
	via the configuration file those services will determine how messages are logged
	and anything passed on the  command line will be ignored.

	The \opendax configuration system is very sophisticated.  It is based on the
	\textit{Lua} scripting language.  In fact all of the configuration files in the
	base \opendax programs are nothing more than Lua scripts.  These scripts run in
	the sandbox of the program that calls them and some global data is pre-defined
	to help.

	Using a scripting language like Lua makes configuration very powerful.  Some of
	the simple things that you can do are using variables to store commonly used
	values, such as a file system path or an IP address that needs to be reused in
	many parts of the configuration.  If there are parts of the configuration that
	are common between many client modules, that information can be stored in single
	files and the function \texttt{dofile()} can be used to import that code into
	many configurations.  A common example of this is the message logging
	configuration.

	Before a configuration script is executed certain 'globals' are assigned to the
	script.  A very important one is the string \texttt{calling\_module}.  This
	string contains the name of the module that is running the configuration script.
	This allows for common code to be shared between different parts of the system
	and simple \texttt{if} statements to be used to separate which code will be run.
	The \texttt{print()} function is overridden in the configuration script to send
	the given string to the message logging facility instead of simply printing to
	the console.  These messages will be sent to any service that has the
	\texttt{CONFIG} topic configured.

	Other functions and globals may be set for different modules.  See the
	individual module's documentation for more information.

	The default configuration file for the tagserver is \texttt{./tagserver.conf}
	and the default configuration filename for the master daemon is
	\texttt{./opendax.conf}.  Typically the default configuration filename for a
	client module is that module's name with \texttt{.conf} extension.  This can
	overridden with the \texttt{-C} command line option.

	\chapter{The OpenDAX Master Server}

    The \opendax{} Master Server is a program that manages all of the
    other programs in the \opendax{} system.  It can be configured to
    start and monitor all of the process that are needed for any
    given application.

    The Master Server is named \texttt{opendax}.

    <<Need much more work here>>

	\chapter{The Tag Server}

    \section{Overview}
	The tag server contains the central database of all the real-time data.  The
	entire system is centered around the tag server.

	The tag server is named \texttt{tagserver} and it is very simple to run
	and will
	typically run just fine without any configuration using the defaults.  There are
	only a few configuration options.

    \section{Configuration}

	\begin{list}{}{Command Line Options}
		\item[\texttt{--configfile, -C}] - This option requires an argument that
		represents the configuration file that will be read on system start.  By default
		the file \texttt{./tagserver.conf} will be used.
		\item[\texttt{--version, -V}] - If this option is set, the tag server will
		simply display version information and exit.
		\item[\texttt{--logtopics, -T}] - This option requires an argument that is a
		comma delimited string containing the topics that we wish to log.  It only
		affects the default logging service.  If any logging services are configured in
		the configuration file then this will be overridden by the newly created logging
		service.
		\item[\texttt{--verbose, -v}] - This option simply sets the default logging
		topics to ALL.  This has the effect of passing all the messages to the default
		logging service.
		\item[\texttt{--socketname, -S}] - This option requires an argument that
		represents the filename that will be used for the local domain socket.  By
		default it's \texttt{/tmp/opendax}.
		\item[\texttt{--serverip, -I}] - The IP address that the server will listen on.
		Default is 0.0.0.0 which has the effect of listening on all interfaces.
		\item[\texttt{--serverport, -P}] - The TCP port that the server will bind to.
		Default is 7777.
		\item[\texttt{--mod-tag-exclude, -X}] - A space delimited list of module names
		that will be excluded from the module tag creation feature.  Useful for modules
		that will start and stop often and the features of the module tag are not needed
	\end{list}

	Some of this configuration is duplicated in the configuration file.  Here is an
	example configuration file.


	\begin{verbatim}
	-- Main configuration file for the OpenDAX tag server

	-- Configuration files that are common for all programs
	dofile(configdir .. "/log.conf")

	-- The name of the local domain socket that will be created for local
	-- client module communications
	socketname = "/tmp/opendax"

	-- The IP address and TCP port that the server will listen on
	serverip = "0.0.0.0"
	serverport = 7777

	-- Module tag exclusion list. This is a space delimited list of module names that
	-- will be excluded from the module tag feature.  This might be useful for modules
	-- that start and stop often (i.e. modules run from shell scripts) or modules that
	-- have no need of the module tag features.
	mod_tag_exclude = "daxc qdax"

	-- The minimum number of communication buffers that will be maintained
	-- in the system.  The server uses pre-allocated buffers for communication
	-- and this designates the minimum number that will be maintained.
	min_buffers = 5

	\end{verbatim}

	The \texttt{dofile()} functions simply load and run the configuration scripts
	that are given.  This is a very powerful feature of OpenDAX configuration.  It
	allows for common configuration code to be reused in many parts of the OpenDAX
	system.  Here we are just putting all of the message logging configurations in a
	separate file.

    \section{Detailed Description}



	\chapter{Modules}
	Modules are where all the real work gets done on an \opendax{} system.  The server
	is the glue that holds all the modules together but the modules are where it all
	happens.  The \opendax{} distribution comes with a few included modules.  They are
	described in this chapter.  If you have received modules from other sources
	you'll have to consult the documentation that came with that module.

	\section{Common Module Configuration}

	The module configuration file is usually named after the module.  For instance
	the configuration file for the \emph{daxlua} module is called
	\emph{daxlua.conf}.  These files would contain information that is specific to
	each module.  The \emph{daxlua} configuration contains the location of the
	script files to execute as an example.

	There are some configuration options that are common to all modules that are
	included in the distribution.  Some can only be given on the command line,
	others only in the configuration file.

	\begin{list}{}{Command Line Options}
		\item[\texttt{--config, -C}] - The full path to the requested configuration
		file.  This file will be used in lieu of the default configuration.
		\item[\texttt{--confdir, -K}] - If this option is specified the default
		configuration directory will be changed.  The normal default configuration
		filename will be used (i.e. <modulename>.conf) but it will be loaded from this
		directory.
		\item[\texttt{--logtopics, -T}] - This option requires an argument that is a
		comma delimited string containing the topics that we wish to log.  It only
		affects the default logging service.  If any logging services are configured in
		the configuration file then this will be overridden by the newly created logging
		service.
		\item[\texttt{--verbose, -v}] - This option simply sets the default logging
		topics to ALL.  This has the effect of passing all the messages to the default
		logging service.
		\item[\texttt{--socketname, -U}] - This option requires an argument that
		represents the filename that will be used for the local domain socket.  By
		default it's \texttt{/tmp/opendax}.
		\item[\texttt{--serverip, -I}] - The IP address that the module will use for
		network connections.  Default it \texttt{127.0.0.1}.
		\item[\texttt{--serverport, -P}] - The TCP port that the module will use for
		network connections.  Default is 7777.
		\item[\texttt{--server, -S}] - Should be either \texttt{LOCAL} or
		\texttt{NETWORK}.  If set to \texttt{NETWORK} a TCP connection will be made to
		the tag server.  Otherwise the unix domain socket will be used.  Default is
		\texttt{LOCAL}
		\item[\texttt{--name, -N}] - Module name.  The name of the module is used for
		determining the default configuration file, it's sent to the server to identify
		the module to the server and it's passed to the modules configuration file as
		the \texttt{calling\_module} global variable.  It's also used in the message
		logging system for some services.
		\item[\texttt{--cachesize, -Z}] - The \texttt{libdax} library maintains a cache of
		recently used tag definitions so that it does not have to constantly query the
		server for this information every time tags need to be accessed.  The higher
		this number is the less the module will have to query the server but the more
		memory will be used.
		\item[\texttt{--msgtimeout, -O}] - The time before a timeout error will be generated for
		communication with the tag server.  Other timeouts may need to be set for
		specific client module communications, but this one is specifically for the
		client module / tag server communications.
	\end{list}

	The following is an example of a module configuration file.

	\begin{verbatim}
	-- Example module configuration file, showing the configuration
	-- options that are common to all the modules supplied with OpenDAX

	-- Configuration files that are common for all programs
	dofile(configdir .. "/log.conf")

	-- The name can be set here and it will effect the name given to the
	-- server once connected as well as any message logging services that
	-- use the module name.  Obviously the configuration file and the
	-- global variable 'calling_module' will be either the default name of
	-- the module or a name that was passed on the command line.
	name = "monkey"

	-- The name of the local domain socket that will be used to
	-- communicate with a local server
	socketname = "/tmp/opendax"

	-- The IP address and TCP port of the server
	serverip = "0.0.0.0"
	serverport = 7777

	-- When set to NETWORK the client module will use TCP/IP as the
	-- communication mechanism to the server.
	server = "LOCAL"
	--server = "NETWORK"

	-- The size of the tag cache
	cachesize = 32

	-- The server message timeout in mSec
	msgtimeout = 500
	\end{verbatim}

	With the exception of the message logging configuration the command line options
	will override the configuration file options.

	\chapter{Command Line Client Module}
	The command line client module is called 'daxc'.  It is useful for debugging and
	examining or manipulating \opendax{} data in real time.  It can also be used for
	automating some \opendax{} tasks from a shell script.  It can be run interactively,
	it can read input from STDIN or a filename can be passed to it that has lists of
	\texttt{daxc} commands to execute.

	\section{Configuration}
	There is very little configuration that needs to be done to the \texttt{daxc}
	module.  The general configuration that is needed for all modules such as server
	address, ports etc are all that is really required.  There are a few command
	line options that are specific to the \texttt{daxc} module.

	There is no configuration file for \texttt{daxc}.  All of the configuration is
	passed on the command line.

	\begin{verbatim}
	Usage:
	$ daxc [-xiqUIPSZOCKTv] [filename]
	\end{verbatim}

	If \texttt{filename} is given the commands in the file will be executed as
	though
	they were received through \texttt{stdin}.  \texttt{daxc} will also act
	correctly if a \textit{shebang} is used for the file.  So the file can also be
	marked as executable and \texttt{daxc} will act as expected.  Blank lines and
	lines that start with '\#' are ignored.

	\begin{list}{}{daxc Command Line Options}
		\item[\texttt{--execute, -x}] - The execute options should be followed by a command
		that is to be executed by \texttt{daxc}.
		\item[\texttt{--interactive, -i}] - When the module is started with the -f or -x
		options the default behavior of the module is to exit when finished with those
		commands.  If this option is used the program will enter interactive mode after
		those commands are issued.
		\item[\texttt{--quiet, -q}] - Suppresses some output of the program while it is
		executing a file or a command line option.  It does nothing for interactive
		mode.  This is useful for running the module from shell scripts.
	\end{list}

	\section{Commands}

	\subsection{list}
	The \textit{list} command lists information about tags and datatypes.  If the
	command is given no arguments it will simply list all of the tags in the system.
	If followed by the subcommand \textit{type} it will list all of the compound
	data types that are configured in the currently running system.  If
	\textit{type} is followed by the name of one of those compound data types it
	will list the members of the cdt and their data type.

	\subsection{cdt}
	\begin{verbatim}
	Usage:
	daxc>cdt name m1name m1type m1count [m2name] [m2type] [m2count] [...]
	\end{verbatim}
	The \textit{cdt} command is used to create a compound data type.  The command
	should be followed by the name of the new data type and then as many member
	triplicates as are needed to define the type.  The triplicates include the name
	of the cdt member the datatype of that member and the count for the member.  For
	example...

	\begin{verbatim}
	daxc>cdt newType myInt INT 1 myDint DINT 10 myBool BOOL 32
	\end{verbatim}

	This would create a new compound datatype with three members.  The first member
	a single INT named myInt the second an array of 10 DINT's named myDint and the
	last one is an array of 32 BOOLs named myBool.

	You can put as many triplicates after the name as needed to define the cdt.  The
	cdt cannot be redefined, once it is created that's it.  It'll be possible to
	delete unused types at some point but that feature is not yet implemented.

	\subsection{add}
	\begin{verbatim}
	Usage:
	daxc>add tagname type count
	\end{verbatim}

	Adds a tag to the system.  The command should be followed by the name of the new
	tag, the data type of the new tag and the count for the size of the tag.

	\subsection{read}
	\begin{verbatim}
	Usage:
	daxc>read tagname [count]
	\end{verbatim}

	The read command should be followed by the tag that is to be read.  The
	count argument can be given if only a subset of the tags are to be read.

	\subsection{write}
	\begin{verbatim}
	Usage:
	daxc>write tagname value 1 [value2] [...]
	\end{verbatim}

	\subsection{db}
	\begin{verbatim}
	Usage:
	daxc>db tagname [count]
	\end{verbatim}

	\subsection{help}
	\begin{verbatim}
	Usage:
	daxc>help [command]
	\end{verbatim}

	The \textit{help} command followed by the name of another command will list
	information about using that command. Otherwise it simply prints a list of
	commands that help is available on.

	\subsection{exit}
	\begin{verbatim}
	Usage:
	daxc>exit
	\end{verbatim}
	The \textit{exit} command simply exits the daxc module.

	\chapter{Modbus Communications Module}
	The Modbus Client module is a very powerful \modbus{} Protocol communication module.
It is capable of Modbus TCP Client and Server as well as Modbus RTU Master and
Slave.

The basic functional unit of the Modbus module is the \textit{port}.  A port is
associated with a single communication mechanism.  Either a TCP socket or a serial
port.  The port can be configured as either a Master/Client or a Server/Slave.
Multiple ports can be configured so that many serial ports can be used or Multiple
servers can be each assigned a different IP address or port number.  There is no
arbitrary limit on the number of ports that can be configured.

When a port is configured as a master or a client, a list of commands can be
configured that will be sent periodically to access the data in the server or
slave node.  A tag can be created as well that allows other modules in the
system to manually send a modbus request on the port.

When the port is configured as a server or slave, tags are set up that represent
the different register tables (holding, inputs, etc).  These tags would either be arrays
of bits in the case of inputs and coils or 16 bit integers in the case of holding
registers or analog inputs.  Multiple sets of these register tags can be configured
so that multiple nodes or units can be represented with different sets of registers.
If a register is not defined for a specific unit
or node then requests to access those registers will cause unknown function code
errors to be returned to the master/client.

There are also hooks in the server/slave ports that call Lua functions at certain
points in the communication.  This allows the user to intercept a message and
return errors or modify the data on the fly or write tags to synchronize
logic with modbus requests.


\section{Configuration}



	\chapter{PLCTag Communication Module}
	The PLCTag Client module utilizes the \textit{libplctag} library for reading
and writing tags to various PLCs.

PLC tags are defined in the plctag.conf configuration file and are associated
with \opendax{} tags.  The PLCs are polled for their respective tags at intervals
that are configured in the configuration file and updated in the \opendax{}
tag server.

\section{Configuration}

\begin{verbatim}
-- Tag definitions are a table that will be passed to
-- the add_plctag() function later.  The name is
-- arbitrary.   We use tag here but you can use any
-- valid Lua identifier
tag = {}

-- daxtag and count define the tag in the OpenDAX tag server
tag.daxtag = "TestUINT" -- Name of the tag in OpenDAX.
tag.count = 5  -- Number of elements in the above tag

-- elem_size overrides the size of the element in the PLC.
tag.elem_size = 4

-- plctag is the attribute definition of the PLC tag that defines the
-- tag in the libplctag library.  See the libplctag documentation for
-- a complete description. 

-- tag.plctag = "protocol=ab-eip&gateway=172.16.4.1&path=1,1&
--               cpu=LGX&elem_count=10&name=TestDINT"
tag.plctag = "protocol=modbus_tcp&gateway=127.0.0.1&path=1&
              elem_size=4&elem_count=10&name=hr0"


tag.read_update = 1000  -- amount of time between reads from the PLC
tag.write_update = 500  -- amount of time between writes to the PLC

-- Finally we create the tag in the module configuration
add_plctag(tag)
\end{verbatim}

The main configuration of the PLCTag module is based around the 
definition of 'tags.'  This can be a little confusing since we
use the name \textit{tag} as the definition of a single unit of
data within \opendax.  Here we differentiate between a \textit{PLC tag} and
an \opendax{} \textit{tag}.

Tags are defined in the configuration with a table that uses
strings as the index.  This table is set up and then passed to
\texttt{add\_tag()} configuration function.  The table can then
be reused or another table can be defined.

A PLC tag is defined with a URL like string that defines the tag
for the libplctag library.  \opendax  simply passes this string to
the libplctag library.  See the libplctag documentation at...

\texttt{https://github.com/libplctag/libplctag/wiki/Tag-String-Attributes}

This string is configured by setting the \texttt{plctag}
member of the tag table in the Lua configuration file.

The \opendax tag is defined with the \texttt{daxtag} member of the
tag table.  The PLCTag module does not create this tag.  It can
be created in an \texttt{init\_hook()} function or in another module.
The tag need not exist when the PLCTag module first starts but errors
will be generated in the log each time the module tries to write the
tag to the tagserver, until the tag is created.

The \texttt{count} member defines the number of items in the \opendax
tag we will use.  This count is the number of items that we will try
to read from the PLC Tag, regardless of how many elements the PLCTag
represents.

The \texttt{elem\_size} member defines the assumed element size of
the PLC tag.  If this is not present or set to zero then the actual
element size will be determined from the libplctag library.  This 
can be overridden to make it possible for the tag in \opendax{} to be
larger than the tag in the PLC.  One use for this would be in the case
where a 32 bit floating point number is being represented with two
consecutive 16 bit numbers in the PLC.  In this case the element size
can be overridden to 4 and this will work as expected.

The \texttt{read\_update} member defines the read polling time
for the tag in milliseconds.  The libplctag library will poll this
tag at that rate and when the tag is read it will be written to the
\opendax{} tag server.

The \texttt{write\_update} member defines the rate in which the writes
to the PLC will happen.  When defined tags change in the tag server an
event is generated in the PLCTag module.  The write is queued in the
libplctag library and will be written to the PLC at the expiration of
the write\_update time.  This allows multiple values to be written to
the PLC at once.  This can make the communication to the PLC much more
efficient since many PLC allow grouping of data in messages, like writing
multiple registers in \modbus{}.

\section{Detailed Description}

Tags are mapped according to a couple of criteria.  The tag data is
read into a buffer within libplctag and then are translated into the 
proper data type for \opendax{}.  The conversion is based on the data
type of the \opendax{} tag.  If the \opendax{} tag is a INT 
(16 bit signed integer) and the PLC tag is of type DINT (32 bit
signed integer) the conversion will work as expected as long as the
value in the PLC is less than or equal to 65536.  At that point the
the value will roll over.  This is the case for single tags.  If the
tags are arrays then it gets more confusing.

The array of data is still read into the libplctag library as a whole
and is represented as a buffer of bytes.  The conversion will still
take place based on the size of the \opendax{} tag but the offset into
the PLC tag buffer will be based on the element size.  This element
size is determined by the PLC tag definition in the library and may be
dependent on the type of PLC as well as the type of tag.  If the number
of elements in the tag and the \textit{count} of the \opendax{} tag are 
the same and the data types are the same then there is no issue.

If the size of the \opendax{} tag is greater than the element size of the
PLC tag then the element size can be overridden in the configuration.
This would allow us to have 8 32 bit floating point numbers represented
in the PLC as 16 16 bit integers or 32 8 bit bytes.  The element count
in the \texttt{plctag} configuration would need to be set appropriately
for this to work correctly.

As an example, let's say that we want to read 8 REAL (32 bit floats) into
\opendax{} from a \modbus{} PLC.  The floats are stored in the modbus PLC
as 16 consecutive registers in the holding register block.  We would
configure the \texttt{plctag} member of the configuration to read these
16 registers and we'd configure the \texttt{daxtag} and \texttt{count}
members as the name of a REAL tag in \opendax{} with a count of 8.  Then
since the normal element size would normally be 2 for the PLC we'll have
override this with a value of 4.  This would work as we expect.

The conversion is always done with the type found in the \opendax{} tag, and
is carried out based on the count found in the \opendax{} tag.  If the count
is left off it will be set to 1.

If the \opendax{} tag is a \textit{Custom Data Type} (CDT) then no data type
conversions will be done and the data will be simply copied byte by byte.  If
the \opendax{} tag is a CDT and the PLC tag is a ControlLogix UDT and they
are structured correctly then it should work.  Testing should be done since
the way PLCs pack structured data and the way \opendax{} packs structured
data may not be the same.  ControlLogix PLCs pack consecutive booleans into
a 32 bit space whereas \opendax{} packs them in 8 bit spaces.  This might mean
that there needs to be 'filler' bytes added to align the rest of the data
type.  The user will have to have a good understanding of the way these
data types are structured on both ends for this to work.

Of course, it is entirely possible to pack data into arrays of bytes in
the PLC that are represented by CDTs in \opendax{} and the reverse is
also true.  Byte ordering is an issue here and we are working on a way
to deal with this problem.  Byte ordering for other types works fine
within libplctag but the raw byte copy that happens for CDTs does not
deal with byte ordering properly.

The offset into the PLC tag buffer is based on the element size and the 
\opendax{} tag count.  Each iteration through the \opendax{} tag
the offset is incremented by the element size.  This is either the size determined
by the PLC tag when it is created or from the \texttt{elem\_size} configuration
if it is given.  It is important that the PLC tag buffer be the same size
or larger than the \opendax{} tag count x the element size.  If not then
overruns will happen.  These overruns are safe since the libplctag library
checks for this condition and doesn't allow unsafe memory accesses but
how this data is returned is not clear for all data types.  Typically
a minimum value is returned but this could change.  Check the \opendax{}
log for these errors if the data you are receiving does not seem correct.

Boolean values work as expected.  If the \opendax{} tag is a BOOL then
the count given by the tag will represent bits.  This will behave as expected
if the PLC tag is a single 32 bit unsigned integer and the \opendax{} tag is
an array of 32 bits.  Each \opendax{} BOOL will represent each bit position
within the PLC integer.

	\chapter{Lua Scripting Module}
	\input{daxlua.tex}

	\chapter{MQTT Module}
	\input{mqtt.tex}

\end{document}

