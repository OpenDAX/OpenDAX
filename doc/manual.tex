\documentclass[10pt,letterpaper]{report}

\usepackage[american]{babel}
\usepackage{amsmath}
\usepackage{amsfonts}
\usepackage{amssymb}
\usepackage{setspace}
\usepackage{parskip}
\usepackage[dvips]{graphicx}
\usepackage{float}
\usepackage{makeidx}
\usepackage{verbatim}
\usepackage{color}
\usepackage{hyperref}
\hypersetup{
    colorlinks=true, % make the links colored
    linkcolor=blue, % color TOC links in blue
    urlcolor=red, % color URLs in red
    linktoc=all % 'all' will create links for everything in the TOC
}

\setlength\parskip{10pt}
\setcounter{secnumdepth}{3}

\makeatletter
\def\thickhrulefill{\leavevmode \leaders \hrule height 1pt\hfill \kern \z@}
\renewcommand{\maketitle}{\begin{titlepage}%
    \let\footnotesize\small
    \let\footnoterule\relax
    \parindent \z@
    \reset@font
    \null\vfil
    \begin{flushleft}
      \huge \@title
    \end{flushleft}
    \par
    \hrule height 1pt
    \par
    \begin{flushright}
      \LARGE \@author \par
    \end{flushright}
    \vskip 60\p@
    \vfil\null
  \end{titlepage}%
  \setcounter{footnote}{0}%
}

\makeatother
\makeindex

\def\opendax{\textit{OpenDAX}}
\def\modbus{\textit{Modbus}$\textsuperscript{\textregistered}$}
\def\daxstate{\texttt{dax\_state} }
\def\eventadd{\texttt{dax\_event\_add()}}
\def\eventdel{\texttt{dax\_event\_del()}}
\def\eventwait{\texttt{dax\_event\_wait()}}
\def\eventpoll{\texttt{dax\_event\_poll()}}
\def\eventgetfd{\texttt{dax\_event\_get\_fd()}}
\def\eventdispatch{\texttt{dax\_event\_dispatch()}}


\title{OpenDAX User's Manual}
\date{June 13, 2008}
\author{Phil Birkelbach}


\begin{document}
	\pagenumbering{roman}
	\maketitle
	
	\begin{flushleft}
		Copyright \textcopyright 2008 - Phil Birkelbach\linebreak
		All Rights Reserved
	\end{flushleft}
	
	\tableofcontents
	\newpage
	\pagenumbering{arabic}
	\chapter{Introduction}
	OpenDAX is an open source, modular, data acquisition and control system
	framework. It is licensed under the GPL (GNU Public License) and therefore is
	completely free to use and modify. OpenDAX is written primarily for Linux. 
	There should be no reason that it wouldn't compile on other *nix like operating
	systems but for now we are concentrating on Linux.
	
	OpenDAX could be used for anything from controlling the air conditioner in a
	home to controlling an entire industrial facility. Depending on what modules are
	installed and run it could be used as the firmware for a dedicated Programable
	Logic Controller (PLC) or a Home Automation system. It could loosely be compared
	to DCS (Distributed Control System) or a SCADA (Supervisory Control and Data
	Acquisition) system. Eventually the system would be able to scale up to a
	several hundred thousand tag system. At this moment the code is far to immature
	to be used for anything that required reliability but we hope to get it to that
	point sooner or later. Much will depend on how many developers take up the
	challenge to help work on the code.
	
	DAX stands for Data Acquisition and eXchange. The system works by combining a
	master program (opendax), a real time database server (tagserver) a library
	(libdax) and set of modules. The opendax server handles the starting, stopping
	and monitoring of process that make up a particular OpenDAX system.  It is
	actually an optional part of the system.  The user can start and stop each
	process manually if desired.
	
	The tag server is the program that manages the real time data that is used
	throughout the system.  It could hold the temperature or pressure of a process,
	the status of a switch or some command data from the Human Interface to a logic
	module.  Each piece of information is called a 'Tag'  Each tag can be a single
	value or an array of values.  (Multi dimensional arrays are not supported.) 
	Each tag is defined as a given data type.  There are several base data types.
	These represent simple number types like INT or FLOAT.  It is possible to create
	compound
	data types that are collections of the base data types and other compound data
	types.
	
	The modules do all the work and communicate with the \texttt{tagserver} through
	the \texttt{libdax} library. There could be modules for reading and writing to
	I/O points, data logging, alarming, machine interface, and logic. The primary
	interface to the tag server is the \texttt{daxc} module, which is a command line
	interface module that can
	be used to read and write tags, check status and do all other types of
	maintenance in the tag server.
	
	There is no requirement that all of the modules have to be on the same machine. 
	There is also no requirement that the modules be on the machine with the tag
	server.  Any process that is to be started by the \texttt{opendax} server will
	have to be on the same computer.  Modules communicate to the server by one of
	two mechanisms.  Either a TCP/IP connection or a \textit{local domain socket}. 
	The \textit{local domain socket} has the advantage of being very fast because
	the operating system kernel simply copies memory from one process to another. 
	The disadvantage is that the module that wants to use this communication
	mechanism must be on the same machine as the tag server. If the module is on
	another machine then it will have to use the TCP/IP protocol to communicate with
	the tag server.
	
	The system is designed so that most of the work happens in the client library
	and the modules.  Since the tag
	server is a bottleneck for the entire system, any work that could be offloaded
	to the client library and the
	modules was.
	
	\chapter{Getting Started}
	\section{Installation}

OpenDAX uses the CMake build system generator.  You'll need to install CMake
on your system.

You will also need the Lua development libraries installed.  Most 
distributions have versions of Lua that will work.  The currently supported
versions of Lua are 5.3 and greater.
If you install Lua from the source files you will need to add -FPIC 
compiler flag to the build.

\begin{verbatim}
make MYCFLAGS="-fPIC" linux
\end{verbatim}

Once you have CMake and the Lua libraries installed you can download and build
OpenDAX.  First clone the repository...

\begin{verbatim}
git clone https://github.com/OpenDAX/OpenDAX.git
\end{verbatim}

This should create the OpenDAX directory.  Now do the following...
\begin{verbatim}
mkdir build
cd build
cmake ..
make
make test
\end{verbatim}

If all the tests pass you can install with \ldots

\begin{verbatim}
sudo make install
sudo ldconfig
\end{verbatim}

	
	\section{Configuration}
	There are two ways to pass configuration information into any given \opendax
	program.  The first is simply to pass command line arguments and the second is
	via a configuration file.  There is likely some overlap between what options can
	be passed with either method but some options only make sense for the command
	line (an example is the \texttt{config} option that specifies which
	configuration file to use).  Some options might be too complex to put on the
	command line and will require a configuration file.
	
	If a configuration attribute is present on the command line it will take
	precedence over that same attribute given in the configuration files.  An
	exception to this is the logging configuration.  Once logging services are added
	via the configuration file those services will determine how messages are logged
	and anything passed on the  command line will be ignored.
	
	The \opendax configuration system is very sophisticated.  It is based on the
	\textit{Lua} scripting language.  In fact all of the configuration files in the
	base \opendax programs are nothing more than Lua scripts.  These scripts run in
	the sandbox of the program that calls them and some global data is pre-defined
	to help.
	
	Using a scripting language like Lua makes configuration very powerful.  Some of
	the simple things that you can do are using variables to store commonly used
	values, such as a file system path or an IP address that needs to be reused in
	many parts of the configuration.  If there are parts of the configuration that
	are common between many client modules, that information can be stored in single
	files and the function \texttt{dofile()} can be used to import that code into
	many configurations.  A common example of this is the message logging
	configuration.
	
	Before a configuration script is executed certain 'globals' are assigned to the
	script.  A very important one is the string \texttt{calling\_module}.  This
	string contains the name of the module that is running the configuration script.
	This allows for common code to be shared between different parts of the system
	and simple \texttt{if} statements to be used to separate which code will be run.
	The \texttt{print()} function is overridden in the configuration script to send
	the given string to the message logging facility instead of simply printing to
	the console.  These messages will be sent to any service that has the
	\texttt{CONFIG} topic configured.
	
	Other functions and globals may be set for different modules.  See the
	individual module's documentation for more information.
	
	The default configuration file for the tagserver is \texttt{./tagserver.conf}
	and the default configuration filename for the master daemon is
	\texttt{./opendax.conf}.  Typically the default configuration filename for a
	client module is that module's name with \texttt{.conf} extension.  This can
	overridden with the \texttt{-C} command line option.
	
	\chapter{The OpenDAX Master Server}
	
	
	\chapter{The Tag Server}
	The tag server contains the central database of all the real-time data.  The
	entire system is centered around the tag server.
	
	The tag server is named \texttt{tagserver} it is very simple to run and will
	typically run just fine without any configuration using the defaults.  There are
	only a few configuration options.
	
	\begin{list}{}{Command Line Options}
		\item[\texttt{--configfile, -C}] - This option requires an argument that
		represents the configuration file that will be read on system start.  By default
		the file \texttt{./tagserver.conf} will be used.
		\item[\texttt{--version, -V}] - If this option is set, the tag server will
		simply display version information and exit.
		\item[\texttt{--logtopics, -T}] - This option requires an argument that is a
		comma delimited string containing the topics that we wish to log.  It only
		affects the default logging service.  If any logging services are configured in
		the configuration file then this will be overridden by the newly created logging
		service.
		\item[\texttt{--verbose, -v}] - This option simply sets the default logging
		topics to ALL.  This has the effect of passing all the messages to the default
		logging service.
		\item[\texttt{--socketname, -S}] - This option requires an argument that
		represents the filename that will be used for the local domain socket.  By
		default it's \texttt{/tmp/opendax}.
		\item[\texttt{--serverip, -I}] - The IP address that the server will listen on.
		Default is 0.0.0.0 which has the effect of listening on all interfaces.
		\item[\texttt{--serverport, -P}] - The TCP port that the server will bind to. 
		Default is 7777.
	\end{list}
	
	Some of this configuration is duplicated in the configuration file.  Here is an
	example configuration file.
	
	
	\begin{verbatim}
	-- Main configuration file for the OpenDAX tag server
	
	-- Configuration files that are common for all programs
	dofile(configdir .. "/log.conf")
	
	-- The name of the local domain socket that will be created for local 
	-- client module communications
	socketname = "/tmp/opendax"
	
	-- The IP address and TCP port that the server will listen on
	serverip = "0.0.0.0"
	serverport = 7777
	
	-- The minimum number of communication buffers that will be maintained
	-- in the system.  The server uses pre-allocated buffers for communication
	-- and this designates the minimum number that will be maintained.
	min_buffers = 5
	
	\end{verbatim}
	
	The \texttt{dofile()} functions simply load and run the configuration scripts
	that are given.  This is a very powerful feature of OpenDAX configuration.  It
	allows for common configuration code to be reused in many parts of the OpenDAX
	system.  Here we are just putting all of the message logging configurations in a
	separate file.
	
	\chapter{Modules}
	Modules are where all the real work gets done on an \opendax system.  The server
	is the glue that holds all the modules together but the modules are where it all
	happens.  The \opendax distribution comes with a few included modules.  They are
	described in this chapter.  If you have received modules from other sources
	you'll have to consult the documentation that came with that module.
	
	\section{Common Module Configuration}
	
	The module configuration file is usually named after the module.  For instance
	the configuration file for the \emph{daxlua} module is called
	\emph{daxlua.conf}.  These files would contain information that is specific to
	each module.  The \emph{daxlua} configuration contains the location of the
	script files to execute as an example.
	
	There are some configuration options that are common to all modules that are
	included in the distribution.  Some can only be given on the command line,
	others only in the configuration file.
	
	\begin{list}{}{Command Line Options}
		\item[\texttt{--config, -C}] - The full path to the requested configuration
		file.  This file will be used in lieu of the default configuration.
		\item[\texttt{--confdir, -K}] - If this option is specified the default
		configuration directory will be changed.  The normal default configuration
		filename will be used (i.e. <modulename>.conf) but it will be loaded from this
		directory.
		\item[\texttt{--logtopics, -T}] - This option requires an argument that is a
		comma delimited string containing the topics that we wish to log.  It only
		affects the default logging service.  If any logging services are configured in
		the configuration file then this will be overridden by the newly created logging
		service.
		\item[\texttt{--verbose, -v}] - This option simply sets the default logging
		topics to ALL.  This has the effect of passing all the messages to the default
		logging service.
		\item[\texttt{--socketname, -U}] - This option requires an argument that
		represents the filename that will be used for the local domain socket.  By
		default it's \texttt{/tmp/opendax}.
		\item[\texttt{--serverip, -I}] - The IP address that the module will use for
		network connections.  Default it \texttt{127.0.0.1}.
		\item[\texttt{--serverport, -P}] - The TCP port that the module will use for
		network connections.  Default is 7777.
		\item[\texttt{--server, -S}] - Should be either \texttt{LOCAL} or
		\texttt{NETWORK}.  If set to \texttt{NETWORK} a TCP connection will be made to
		the tag server.  Otherwise the unix domain socket will be used.  Default is
		\texttt{LOCAL}
		\item[--name, -N] - Module name.  The name of the module is used for
		determining the default configuration file, it's sent to the server to identify
		the module to the server and it's passed to the modules configuration file as
		the \texttt{calling\_module} global variable.  It's also used in the message
		logging system for some services.
		\item[--cachesize, -Z] - The \texttt{libdax} library maintains a cache of
		recently used tag definitions so that it does not have to constantly query the
		server for this information every time tags need to be accessed.  The higher
		this number is the less the module will have to query the server but the more
		memory will be used.
		\item[--msgtimeout, -O] - The time before a timeout error will be generated for
		communication with the tag server.  Other timeouts may need to be set for
		specific client module communications, but this one is specifically for the
		client module / tag server communications.
	\end{list}
	
	The following is an example of a module configuration file.
	
	\begin{verbatim}
	-- Example module configuration file, showing the configuration
	-- options that are common to all the modules supplied with OpenDAX
	
	-- Configuration files that are common for all programs
	dofile(configdir .. "/log.conf")
	
	-- The name can be set here and it will effect the name given to the
	-- server once connected as well as any message logging services that
	-- use the module name.  Obviously the configuration file and the
	-- global variable 'calling_module' will be either the default name of
	-- the module or a name that was passed on the command line.
	name = "monkey"
	
	-- The name of the local domain socket that will be used to
	-- communicate with a local server
	socketname = "/tmp/opendax"
	
	-- The IP address and TCP port of the server
	serverip = "0.0.0.0"
	serverport = 7777
	
	-- When set to NETWORK the client module will use TCP/IP as the
	-- communication mechanism to the server.
	server = "LOCAL"
	--server = "NETWORK"
	
	-- The size of the tag cache
	cachesize = 32
	
	-- The server message timeout in mSec
	msgtimeout = 500
	\end{verbatim}
	
	With the exception of the message logging configuration the command line options
	will override the configuration file options.
	
	\chapter{Command Line Client Module}
	The command line client module is called 'daxc'.  It is useful for debugging and
	examining or manipulating OpenDAX data in real time.  It can also be used for
	automating some OpenDAX tasks from a shell script.  It can be run interactively,
	it can read input from STDIN or a filename can be passed to it that has lists of
	daxc commands to execute.
	
	\section{Configuration}
	There is very little configuration that needs to be done to the \texttt{daxc}
	module.  The general configuration that is needed for all modules such as server
	address, ports etc are all that is really required.  There are a few command
	line options that are specific to the \texttt{daxc} module.
	
	There is no configuration file for \texttt{daxc}.  All of the configuration is
	passed on the command line.
	
	\begin{verbatim}
	Usage:
	$ daxc [-xiqUIPSZOCKTv] [filename]
	\end{verbatim}
	
	If \texttt{filename} is given the commands in the file will be executed as
	though
	they were received through \texttt{stdin}.  \texttt{daxc} will also act
	correctly if a \textit{shebang} is used for the file.  So the file can also be
	marked as executable and \texttt{daxc} will act as expected.  Blank lines and
	lines that start with '\#' are ignored.
	
	\begin{list}{}{daxc Command Line Options}
		\item[--execute, -x] - The execute options should be followed by a command
		that is to be executed by daxc.
		\item[--interactive, -i] - When the module is started with the -f or -x
		options the default behavior of the module is to exit when finished with those
		commands.  If this option is used the program will enter interactive mode after
		those commands are issued.
		\item[--quiet, -q] - Suppresses some output of the program while it is
		executing a file or a command line option.  It does nothing for interactive
		mode.  This is useful for running the module from shell scripts. 
	\end{list}
	
	\section{Commands}
	
	\subsection{list}
	The \textit{list} command lists information about tags and datatypes.  If the
	command is given no arguments it will simply list all of the tags in the system.
	If followed by the subcommand \textit{type} it will list all of the compound
	data types that are configured in the currently running system.  If
	\textit{type} is followed by the name of one of those compound data types it
	will list the members of the cdt and their data type.
	
	\subsection{cdt}
	\begin{verbatim}
	Usage:
	daxc>cdt name m1name m1type m1count [m2name] [m2type] [m2count] [...]
	\end{verbatim}
	The \textit{cdt} command is used to create a compound data type.  The command
	should be followed by the name of the new data type and then as many member
	triplicates as are needed to define the type.  The triplicates include the name
	of the cdt member the datatype of that member and the count for the member.  For
	example...
	
	\begin{verbatim}
	daxc>cdt newType myInt INT 1 myDint DINT 10 myBool BOOL 32
	\end{verbatim}
	
	This would create a new compound datatype with three members.  The first member
	a single INT named myInt the second an array of 10 DINT's named myDint and the
	last one is an array of 32 BOOLs named myBool.
	
	You can put as many triplicates after the name as needed to define the cdt.  The
	cdt cannot be redefined, once it is created that's it.  It'll be possible to
	delete unused types at some point but that feature is not yet implemented.
	
	\subsection{add}
	\begin{verbatim}
	Usage:
	daxc>add tagname type count
	\end{verbatim}
	
	Adds a tag to the system.  The command should be followed by the name of the new
	tag, the data type of the new tag and the count for the size of the tag.
	
	\subsection{read}
	\begin{verbatim}
	Usage:
	daxc>read tagname [count]
	\end{verbatim}
	
	The read command should be followed by the tag that is to be read.  At this
	point in time the tag should be a base data type tag.  This means that if you
	enter the name of a tag that resolves to a CDT, then read will fail with an
	error.  At some point the ability to read the entire tag will be implemented but
	at this point the string passed to read should resolve to a base type.  The
	count argument can be given if only a subset of the tags are to be read.
	
	\subsection{write}
	\begin{verbatim}
	Usage:
	daxc>write tagname value 1 [value2] [...]
	\end{verbatim}
	
	\subsection{db}
	\begin{verbatim}
	Usage:
	daxc>db tagname [count]
	\end{verbatim}
	
	\subsection{help}
	\begin{verbatim}
	Usage:
	daxc>help [command]
	\end{verbatim}
	
	The \textit{help} command followed by the name of another command will list
	information about using that command. Otherwise it simply prints a list of
	commands that help is available on.
	
	\subsection{exit}
	\begin{verbatim}
	Usage:
	daxc>exit
	\end{verbatim}
	The \textit{exit} command simply exits the daxc module.
	
	\chapter{Modbus Communications Module}
	
	\chapter{Lua Scripting Module}
	
	\chapter{MQTT Module}
	The MQTT Client module allows the user to connect to an MQTT broker and associate tags with topics.

\opendax tags can be associated with either a publisher or a subscriber and Lua functions are used to format the data from the broker to the \opendax tags.  These formatting functions have the ability to add, read and write tags so the mechanism is very powerful.

\section{Subscriptions}

Subscriptions can subscribe to any topic that is allowed by MQTT.  Wildcards can be used.  The actual topic that is received will be passed to the formatting function that are configured for the subscription so the formatting function can make decisions based on the actual received topic.

The formatting functions have the ability to create, read and write any tag, regardless of whether they are configured along with the subscription.  Tags that are configured in the subscription are more efficient because a buffer is pre-allocated to hold the data and the tag handles are stored.

If no tags are associated with the subscription then it's assumed that the formatting function will write the tag data to the server.  It is not required that the formatting function write any data to the tag server.  It may be that the formatting function decides based on the data to make no changes or the formatting function may do nothing other than write the data to the message logger with the Lua \texttt{print}\footnote{The print function is overridden by the module and will pass the given string to the message logger with the topic of ERR\_LOGIC} function.

When a formatting function and tags are assigned, the module expects the formatting function to return values that will then be converted and assigned to the tags.  It is important that the function return the proper type and number of values that match the configured tags.  Warnings will be written to the message logger if the number of tags configured does not match the number of values returned.

If more than one tag is associated with the subscription then a tag data group is used for the communication to the tag server.  Otherwise, a simple \texttt{tag\_write} is used.  Any tag can be associated with the subscription including arrays and tags that have compound data types.  If subsets of arrays or parts of tags are to be written then a Lua formatting function that directly writes the tags will have to be used.

The simplest (and fastest) formatting method happens when no formatting function is specified.  If no formatter is specified then a simple raw copy of the MQTT message payload is copied to the tag.  Byte ordering issues are ignored and no attempt is made to check the data.  This is extremely fast but obviously not very flexible.  Mirroring a set of tags between two \opendax systems would be very efficient if done this way.

\section{Publishing}

Publishing is similar except that the formatting function should return a single string that will be the data to be published.  If tags are given they will be read and passed to the formatting function.  If no formatting function is given then a raw copy of the tag data will be published to the topic.  At least one tag must be configured to use as a trigger.  The trigger is what causes the publication to happen.  A specific trigger tag can be given.  This will be used to setup a regular dax event and hits on that event will cause the publication.  The type of trigger can also be set and if necessary, a value for those event types that require a value (i.e. Greater Than).  The trigger tag need not be one of the tags that will be part of the publication.  If a trigger tag is not explicitly configured then the first tag that is assigned to the publisher will be the trigger.  If not type is given then the WRITE event type will be used.  If no trigger tag is given and no tags are assigned, an error is raised and the publisher will not be enabled.

The formatting function accepts two parameters, the topic name and the tag values.  If there is a single tag configured for the publisher then the value of that tag will be passed to the function.  If there are more than one tag then an array of values corresponding to the tags will be passed as the value.  If not tags are configured nil will be passed.  It is expected that the formatting function will return a string.  The string can be a binary string that contains zeros.  The \texttt{struct.pack} function may be very useful here.  Also the \texttt{string.format} function.
	
	\chapter{Module Development}
	
\end{document}

