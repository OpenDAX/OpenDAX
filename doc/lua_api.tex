\chapter{Lua API Reference}
\section{Introduction}
The Lua API is a wrapper around the \textit{libdax} library for use in Lua scripts.  Lua scripts can either be run by the \textit{daxlua} module or they can run as stand alone Lua scripts.

If the script is run from the \textit{daxlua} the functions will have bee loaded and the connection to the tag server will have already been made and initialized.

If the script is run as a stand alone interpreter then the \textit{dax} module will have to be loaded and the connection to the server initalized.

\section{General}

\begin{verbatim}
dax.init(modulename)
\end{verbatim}
\index{dax.init function}
Initialize the internal library and attempt to connect to the tagserver.  The function takes one arguement, \textit{modulename}, which is a string that represents the name that will be used to register this module with the server.

This function is only needed for stand alone Lua client modules.  For scripts that are run within the \textit{daxlua} module this initialization has already been done.

This function returns nothing and will raise errors if something fails.
\begin{verbatim}
dax.free()
\end{verbatim}
\index{dax.free function}
Disconnects from the server and frees the internal library data structures.

This function is only needed for stand alone Lua client modules.  For scripts that are run within the \textit{daxlua} module this initialization has already been done.

This function returns nothing and should never fail.
\begin{verbatim}
dax.cdt_create(typename, cdt_table)
\end{verbatim}
\index{dax.cdt\_create function}
The function is used to create a \textit{Compound Datatype}.  The first argument should be a string that will be used as the name of the CDT.  The second argument is a table of tables that defines the members of the CDT.

\begin{verbatim}
members = {{"Name", "DataType", count},
           {"AnotherNmae", "DataType", count}}
\end{verbatim}

This function raises erros on failure and returns a single integer that represents the datatype and can be used to create tags.

\begin{verbatim}
dax.tag_add(name, type, <count>)
\end{verbatim}
\index{dax.tag\_add function}

Adds a tag to the tagserver database.  The first argument is a string that represents the name of the new tag.  The second argument can either be an integer or a string that represents the data type of the tag.  The third argument is represents the number of items created for the datatype.  If this number is greater than 1 then an array is created.  If the count is not given then 1 is assumed.

This function returns nothing on success and raises errors otherwise.

\begin{verbatim}
dax.tag_get(tag)
\end{verbatim}
\index{dax.tag\_get function}
Retrieve the definition of the given tag.  The function takes a single argument that can either be the tagname as a string or the tag index as an integer.

The function returns three values that represent the tag.  They are name, type and count.

\begin{verbatim}
dax.tag_read(tag, <count>)
\end{verbatim}
\index{dax.tag\_write function}
Read and return the value(s) of the given tag.  The first argument is a string representing the tag that we wish to read.  The second, optional, argument is the number of members that we want to read.

The return value depends the type and size of the tag.

\begin{verbatim}
dax.event_add(tagname, count, type, data, callback, userdata)
\end{verbatim}
\index{dax.event\_add function}
Add an event to the system.  The function takes six arguments.

tagname - The first argument is the tagname of the tag that we wish to monitor with this event.

count - The number of items in the tag we wish to monitor.

type - A string representing the event type.  i.e. WRITE, CHANGE, LESS

data - A number that is used for the comparison events EQUAL, GREATER and LESS

callback - The function that we wish to be called when the event is triggered

userdata - This data will be stored by the system and passed to the callback function when this event is triggered

On sucess the function returns a table that represents this event.  On failure errors are raised.

\begin{verbatim}
dax.event_del(event)
\end{verbatim}
\index{dax.event\_del function}
Removes the event from the system.  The single argument to this function is the table that was returned by the \textit{event\_add} function.  

This function returns nothing on success and raises errors on failure.

\begin{verbatim}
dax.event_wait(<timeout>)
\end{verbatim}
\index{dax.event\_wait function}
Blocks and waits for any event to happen and then dispatches that event.  Timeout represents how long the fuction will wait for an event.  If timeout is not given or is set to 0 the function will block forever.

Only a single event will be handled by this function.

The function returns 0 on timeout and 1 if an event was dispatched.

\begin{verbatim}
dax.event_poll()
\end{verbatim}
\index{dax.event\_poll function}
Similar to \textit{event\_wait} except that it does not wait.  If there is an event that is ready to be handled then it will be handled and 1 is returned.  Otherwise it returns 0 immediately.