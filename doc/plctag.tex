The PLCTag Client module utilizes the \textit{libplctag} library for reading
and writing tags to various PLCs.

PLC tags are defined in the plctag.conf configuration file and are associated
with \opendax{} tags.  The PLCs are polled for their respective tags at intervals
that are configured in the configuration file and updated in the \opendax{}
tag server.

\section{Configuration}

\begin{verbatim}
-- Tag definitions are a table that will be passed to
-- the add_plctag() function later.  The name is
-- arbitrary.   We use tag here but you can use any
-- valid Lua identifier
tag = {}

-- daxtag and count define the tag in the OpenDAX tag server
tag.daxtag = "TestUINT" -- Name of the tag in OpenDAX.
tag.count = 5  -- Number of elements in the above tag

-- elem_size overrides the size of the element in the PLC.
tag.elem_size = 4

-- plctag is the attribute definition of the PLC tag that defines the
-- tag in the libplctag library.  See the libplctag documentation for
-- a complete description. 

-- tag.plctag = "protocol=ab-eip&gateway=172.16.4.1&path=1,1&
--               cpu=LGX&elem_count=10&name=TestDINT"
tag.plctag = "protocol=modbus_tcp&gateway=127.0.0.1&path=1&
              elem_size=4&elem_count=10&name=hr0"


tag.read_update = 1000  -- amount of time between reads from the PLC
tag.write_update = 500  -- amount of time between writes to the PLC

-- Finally we create the tag in the module configuration
add_plctag(tag)
\end{verbatim}

The main configuration of the PLCTag module is based around the 
definition of 'tags.'  This can be a little confusing since we
use the name \textit{tag} as the definition of a single unit of
data within \opendax.  Here we differentiate between a \textit{PLC tag} and
an \opendax{} \textit{tag}.

Tags are defined in the configuration with a table that uses
strings as the index.  This table is set up and then passed to
\texttt{add\_tag()} configuration function.  The table can then
be reused or another table can be defined.

A PLC tag is defined with a URL like string that defines the tag
for the libplctag library.  \opendax  simply passes this string to
the libplctag library.  See the libplctag documentation at...

\texttt{https://github.com/libplctag/libplctag/wiki/Tag-String-Attributes}

This string is configured by setting the \texttt{plctag}
member of the tag table in the Lua configuration file.

The \opendax tag is defined with the \texttt{daxtag} member of the
tag table.  The PLCTag module does not create this tag.  It can
be created in an \texttt{init\_hook()} function or in another module.
The tag need not exist when the PLCTag module first starts but errors
will be generated in the log each time the module tries to write the
tag to the tagserver, until the tag is created.

The \texttt{count} member defines the number of items in the \opendax
tag we will use.  This count is the number of items that we will try
to read from the PLC Tag, regardless of how many elements the PLCTag
represents.

The \texttt{elem\_size} member defines the assumed element size of
the PLC tag.  If this is not present or set to zero then the actual
element size will be determined from the libplctag library.  This 
can be overridden to make it possible for the tag in \opendax{} to be
larger than the tag in the PLC.  One use for this would be in the case
where a 32 bit floating point number is being represented with two
consecutive 16 bit numbers in the PLC.  In this case the element size
can be overridden to 4 and this will work as expected.

The \texttt{read\_update} member defines the read polling time
for the tag in milliseconds.  The libplctag library will poll this
tag at that rate and when the tag is read it will be written to the
\opendax{} tag server.

The \texttt{write\_update} member defines the rate in which the writes
to the PLC will happen.  When defined tags change in the tag server an
event is generated in the PLCTag module.  The write is queued in the
libplctag library and will be written to the PLC at the expiration of
the write\_update time.  This allows multiple values to be written to
the PLC at once.  This can make the communication to the PLC much more
efficient since many PLC allow grouping of data in messages, like writing
multiple registers in \modbus{}.

\section{Detailed Description}

Tags are mapped according to a couple of criteria.  The tag data is
read into a buffer within libplctag and then are translated into the 
proper data type for \opendax{}.  The conversion is based on the data
type of the \opendax{} tag.  If the \opendax{} tag is a INT 
(16 bit signed integer) and the PLC tag is of type DINT (32 bit
signed integer) the conversion will work as expected as long as the
value in the PLC is less than or equal to 65536.  At that point the
the value will roll over.  This is the case for single tags.  If the
tags are arrays then it gets more confusing.

The array of data is still read into the libplctag library as a whole
and is represented as a buffer of bytes.  The conversion will still
take place based on the size of the \opendax{} tag but the offset into
the PLC tag buffer will be based on the element size.  This element
size is determined by the PLC tag definition in the library and may be
dependent on the type of PLC as well as the type of tag.  If the number
of elements in the tag and the \textit{count} of the \opendax{} tag are 
the same and the data types are the same then there is no issue.

If the size of the \opendax{} tag is greater than the element size of the
PLC tag then the element size can be overridden in the configuration.
This would allow us to have 8 32 bit floating point numbers represented
in the PLC as 16 16 bit integers or 32 8 bit bytes.  The element count
in the \texttt{plctag} configuration would need to be set appropriately
for this to work correctly.

As an example, let's say that we want to read 8 REAL (32 bit floats) into
\opendax{} from a \modbus{} PLC.  The floats are stored in the modbus PLC
as 16 consecutive registers in the holding register block.  We would
configure the \texttt{plctag} member of the configuration to read these
16 registers and we'd configure the \texttt{daxtag} and \texttt{count}
members as the name of a REAL tag in \opendax{} with a count of 8.  Then
since the normal element size would normally be 2 for the PLC we'll have
override this with a value of 4.  This would work as we expect.

The conversion is always done with the type found in the \opendax{} tag, and
is carried out based on the count found in the \opendax{} tag.  If the count
is left off it will be set to 1.

If the \opendax{} tag is a \textit{Custom Data Type} (CDT) then no data type
conversions will be done and the data will be simply copied byte by byte.  If
the \opendax{} tag is a CDT and the PLC tag is a ControlLogix UDT and they
are structured correctly then it should work.  Testing should be done since
the way PLCs pack structured data and the way \opendax{} packs structured
data may not be the same.  ControlLogix PLCs pack consecutive booleans into
a 32 bit space whereas \opendax{} packs them in 8 bit spaces.  This might mean
that there needs to be 'filler' bytes added to align the rest of the data
type.  The user will have to have a good understanding of the way these
data types are structured on both ends for this to work.

Of course, it is entirely possible to pack data into arrays of bytes in
the PLC that are represented by CDTs in \opendax{} and the reverse is
also true.  Byte ordering is an issue here and we are working on a way
to deal with this problem.  Byte ordering for other types works fine
within libplctag but the raw byte copy that happens for CDTs does not
deal with byte ordering properly.

The offset into the PLC tag buffer is based on the element size and the 
\opendax{} tag count.  Each iteration through the \opendax{} tag
the offset is incremented by the element size.  This is either the size determined
by the PLC tag when it is created or from the \texttt{elem\_size} configuration
if it is given.  It is important that the PLC tag buffer be the same size
or larger than the \opendax{} tag count x the element size.  If not then
overruns will happen.  These overruns are safe since the libplctag library
checks for this condition and doesn't allow unsafe memory accesses but
how this data is returned is not clear for all data types.  Typically
a minimum value is returned but this could change.  Check the \opendax{}
log for these errors if the data you are receiving does not seem correct.

Boolean values work as expected.  If the \opendax{} tag is a BOOL then
the count given by the tag will represent bits.  This will behave as expected
if the PLC tag is a single 32 bit unsigned integer and the \opendax{} tag is
an array of 32 bits.  Each \opendax{} BOOL will represent each bit position
within the PLC integer.