The MQTT Client module allows the user to connect to an MQTT broker and
associate tags with topics.

\opendax tags can be associated with either a publisher or a subscriber and Lua
functions are used to filter the data from the broker to the \opendax tags. 
These filtering functions have the ability to add, read and write tags so the
mechanism is very powerful.

\section{Subscriptions}

Subscriptions can subscribe to any topic that is allowed by MQTT.  Wildcards can
be used.  The actual topic that is received will be passed to the filtering
function that are configured for the subscription so the filtering function can
make decisions based on the actual received topic.

The filtering functions have the ability to create, read and write any tag,
regardless of whether they are configured along with the subscription.  Tags
that are configured in the subscription are more efficient because a buffer is
pre-allocated to hold the data and the tag handles.

If no tags are associated with the subscription then it's assumed that the
filtering function will write the tag data to the server.  It is not required
that the filtering function write any data to the tag server.  It may be that
the filtering function decides based on the data to make no changes or the
filtering function may do nothing other than write the data to the message
logger with the Lua \texttt{print}\footnote{The print function is overridden by
	the module and will pass the given string to the message logger with the topic
	of ERR\_LOGIC} function.

When a filtering function and tags are assigned, the module expects the
filtering function to return values that will then be converted and assigned to
the tags.  It is important that the function return the proper type and number
of values that match the configured tags.  Warnings will be written to the
message logger if the number of tags configured does not match the number of
values returned.

If more than one tag is associated with the subscription then a tag data group
is used for the communication to the tag server.  Otherwise, a simple
\texttt{tag\_write} is used.  Any tag can be associated with the subscription
including arrays and tags that have compound data types.  If subsets of arrays
or parts of tags are to be written then a Lua filtering function that directly
writes the tags will have to be used.

The simplest (and fastest) filtering method happens when no filtering function
is specified.  If no filterter is specified then a simple raw copy of the MQTT
message payload is copied to the tag.  Byte ordering issues are ignored and no
attempt is made to check the data.  This is extremely fast but obviously not
very flexible.  Mirroring a set of tags between two \opendax systems would be
very efficient if done this way.

\section{Publishing}

Publishing is similar except that the filtering function should return a single
string that will be the data to be published.  If tags are given they will be
read and passed to the filtering function.  If no filtering function is given
then a raw copy of the tag data will be published to the topic.  At least one
tag must be configured to use as a trigger.  The trigger is what causes the
publication to happen.  A specific trigger tag can be given.  This will be used
to setup a regular dax event and hits on that event will cause the publication. 
The type of trigger can also be set and if necessary, a value for those event
types that require a value (i.e. Greater Than).  The trigger tag need not be one
of the tags that will be part of the publication.  If a trigger tag is not
explicitly configured then the first tag that is assigned to the publisher will
be the trigger.  If not type is given then the WRITE event type will be used. 
If no trigger tag is given and no tags are assigned, an error is raised and the
publisher will not be enabled.

The filtering function accepts two parameters, the topic name and the tag
values.  If there is a single tag configured for the publisher then the value of
that tag will be passed to the function.  If there are more than one tag then an
array of values corresponding to the tags will be passed as the value.  If not
tags are configured nil will be passed.  It is expected that the filtering
function will return a string.  The string can be a binary string that contains
zeros.  The \texttt{struct.pack} function may be very useful here.  Also the
\texttt{string.filter} function.
