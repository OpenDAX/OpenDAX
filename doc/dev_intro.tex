\chapter{Introduction}
\opendax is an open source, modular, data acquisition and control system. It is licensed under the GPL (GNU General Public License) and therefore is completely free to use and modify.

This book is a tutorial for developing modules for OpenDAX as well as a reference for experienced module developers.  It should be noted that, as of this writing, OpenDAX is very immature and much of the information in this book may have changed.  The developers attempt to keep the interface constant, but at this stage of development it will make sense from time to time to make changes that would break existing module code as well as conflict with information in this book.  The current source code is the ultimate authority on the API.


\section{OpenDAX Basics}
Before we get started, it would be good to discuss how OpenDAX works.  OpenDAX is made up of three main parts, the tag server, the library and the modules.

The OpenDAX tag server\index{OpenDAX server} is the heart and soul of the OpenDAX system.  It sits at the center of the OpenDAX universe and coordinates all of the data and communications of the system.  The libdax library\index{libdax library} abstracts the communication interface to the modules and the modules are where all the work is done.

If the server is the heart and soul of the OpenDAX application then modules are the arms and legs.  Modules do all of the work.  They are separate processes.  Modern operating systems do a great job of managing processes and the OpenDAX developers did not see any need to reinvent that wheel.  

Client Modules handle all of interface to the outside world.  The application logic, any storage or logging functions and the human interface are all handled by modules.  The modules all communicate to the OpenDAX server through an API that is exposed in the libdax library.  The libdax library API is what the OpenDAX module developer will see of OpenDAX.

The low level communications to the OpenDAX server takes place through a BSD Socket interface.  There is no requirement that the module be running on the same machine as the server.  The entire application can be distributed however the application developer desires.  If the module is on the same machine, it can communicate to the server via UNIX Domain Socket.  If the module is on a remote host then it will have to use a TCP Socket.  The UNIX Domain sockets are considerably faster than TCP since they are really nothing more than a memory copy within the kernel.

The exact nature of the communication is subject to change at this point so I
won't go into that in too much detail here

At some point in the future we hope to implement some server-to-server communication that might facilitate redundancy or an even better ability to distribute the system.  Right now there is only one server in any application and all the modules communicate to that single server.

The server contains the tag database.  A tag\index{tag} is the atomic unit of data in the system.  These tags are analogous to variables in a programming language.  There are many different data types in OpenDAX and the user or module programmer can create compound data types that are collections of other types.  Compound data types are similar to a structure in C.  We will discuss compound data types later in this text.

The tag database contains the names of these tags, their data type\index{data type}, the actual real time value and the events that the tag responds too.  It is the central store house of information in OpenDAX.  Different modules work with tags in different ways.  For instance, the \modbus \index{Modbus}module reads data from one or more \modbus devices and stores that information in tags within the server.  The tags are arrays of type UINT or BOOL\footnote{UINT is an unsigned 16 bit data type, BOOL is a single bit data type} depending on the command.  The names of these tags are a configurable parameter of the \modbus module.  If and HMI or Logic module need the information from the \modbus module they would read these tags to get it.

The server also contains information about the currently connected modules. 
Modules must register with the server when they are started before the server
will answer any queries by the module.  This registration step is how the module
and the server learn what they need to learn about each other to properly
communicate.  Once registered the server can keep track of that module through
the file descriptor of the socket on which the the connection is made.  This
information can be made available to other modules that may need it.  All of
this information is made available to the system via system tags that are
created when the module connects and deleted when the module disconnects.

There is another program that is included in the OpenDAX distribution that is optional for the system to function but it's very useful.  It is the master program.  (This is the program that we actuall call 'opendax')  Modules can be started by the OpenDAX master or by any other means that the operating system has for starting processes (i.e. shell prompt, scripts etc).  If the module is to be started by the master there are a few advantages.  First the master will know if the module dies for some reason because the operating system will send it a signal.  This allows the master to restart the module or alert the user that a part of the system is down.

\section{Installation}

OpenDAX uses the CMake build system generator.  You'll need to install CMake
on your system.

You will also need the Lua development libraries installed.  Most 
distributions have versions of Lua that will work.  The currently supported
versions of Lua are 5.3 and greater.
If you install Lua from the source files you will need to add -FPIC 
compiler flag to the build.

\begin{verbatim}
make MYCFLAGS="-fPIC" linux
\end{verbatim}

Once you have CMake and the Lua libraries installed you can download and build
OpenDAX.  First clone the repository...

\begin{verbatim}
git clone https://github.com/OpenDAX/OpenDAX.git
\end{verbatim}

This should create the OpenDAX directory.  Now do the following...
\begin{verbatim}
mkdir build
cd build
cmake ..
make
make test
\end{verbatim}

If all the tests pass you can install with \ldots

\begin{verbatim}
sudo make install
sudo ldconfig
\end{verbatim}


\section{Setting up the Build Environment}
The build environment for developing modules is pretty simple.  If you have installed OpenDAX then you should have everything that you need to compile and run OpenDAX modules.

There is no separate development package for \opendax.  All the files that you need to develop modules should be installed when you install \opendax from the distribution.

The library should be installed in a typical place (usually /usr/local/lib) and the \textit{opendax.h}\index{opendax.h} header file should be in a place where your compiler can find it (usually /usr/local/include).  For writing a module in C this should be all that you need.  If you have problems with the installation, see the \emph{OpenDAX User's Manual}, it has much more detailed information on what is needed to get OpenDAX up and running on your system.

You should be able to use any C compiler to build modules for OpenDAX, but we have been using GCC for the main development.

For all intents and purposes your modules sole interaction with the \opendax system is through the libdax library.  You should include the \emph{opendax.h} header file in your module's source code file and you should link the libdax library with your module with the \verb|-ldax| option to the compiler.\footnote{This is for \texttt{gcc}. Other compilers may have different options for linking shared libraries.}

Lua and Python bingings also exist for OpenDAX.  The Lua library is distributed with OpenDAX and the Python bindings called \textit{PyDax} are in a separate repository.
