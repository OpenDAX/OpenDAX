The Modbus Client module is a very powerful \modbus{} Protocol communication module.
It is capable of Modbus TCP Client and Server as well as Modbus RTU Master and
Slave.

The basic functional unit of the Modbus module is the \textit{port}.  A port is
associated with a single communication mechanism.  Either a TCP socket or a serial
port.  The port can be configured as either a Master/Client or a Server/Slave.
Multiple ports can be configured so that many serial ports can be used or Multiple
servers can be each assigned a different IP address or port number.  There is no
arbitrary limit on the number of ports that can be configured.

When a port is configured as a master or a client, a list of commands can be
configured that will be sent periodically to access the data in the server or
slave node.  A tag can be created as well that allows other modules in the
system to manually send a modbus request on the port.

When the port is configured as a server or slave, tags are set up that represent
the different register tables (holding, inputs, etc).  These tags would either be arrays
of bits in the case of inputs and coils or 16 bit integers in the case of holding
registers or analog inputs.  Multiple sets of these register tags can be configured
so that multiple nodes or units can be represented with different sets of registers.
If a register is not defined for a specific unit
or node then requests to access those registers will cause unknown function code
errors to be returned to the master/client.

There are also hooks in the server/slave ports that call Lua functions at certain
points in the communication.  This allows the user to intercept a message and
return errors or modify the data on the fly or write tags to synchronize
logic with modbus requests.


\section{Configuration}

