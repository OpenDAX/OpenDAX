The Modbus Client module is a very powerful \modbus{} Protocol communication module.
It is capable of Modbus TCP Client and Server as well as Modbus RTU Master and
Slave.

The basic functional unit of the Modbus module is the \textit{port}.  A port is
associated with a single communication mechanism.  Either a TCP socket or a serial
port.  The port can be configured as either a Master/Client or a Server/Slave.
Multiple ports can be configured so that many serial ports can be used or Multiple
servers can be each assigned a different IP address or port number.  There is no
arbitrary limit on the number of ports that can be configured.

When a port is configured as a master or a client, a list of commands can be
configured that will be sent periodically to access the data in the server or
slave node.  A tag can be created as well that allows other modules in the
system to manually send a Modbus request on the port.

When the port is configured as a server or slave, tags are set up that represent
the different register tables (holding, inputs, etc).  These tags would either be arrays
of bits in the case of inputs and coils or 16 bit integers in the case of holding
registers or analog inputs.  Multiple sets of these register tags can be configured
so that multiple nodes or units can be represented with different sets of registers.
If a register is not defined for a specific unit
or node then requests to access those registers will cause unknown function code
errors to be returned to the master/client.

There are also hooks in the server/slave ports that call Lua functions at certain
points in the communication.  This allows the user to intercept a message and
return errors or modify the data on the fly or write tags to synchronize
logic with Modbus requests.


\section{Configuration}

Example configuration files are given in the source code and are pretty self-explanatory.
Essentially, the idea is that one or more ports are created, and then the port id
that is returned from the \texttt{add\_port()} function is used to assign register
definitions or commands to that port, depending on the type of port (master/client vs. slave/server).

If the port is configured as a server or a slave then registers will have to be set up.
The daxmodbus module can act as multiple nodes or units if configured.  Adding a register basically assigns a Dax tag to a particular node/register.  For example you
could assign the tag \textit{mb\_holding} to be the holding register for node 1.  Then all
holding register communications will be reflected in \textit{mb\_holding}.

Each node can be assigned one tag for each register that is understood by Modbus.  One each of the following... Coils, Discrete Inputs, Holding Registers and Analog Input Registers.

If the port is configured as a master or client then commands may have to be set up.  Each command can be set to one of three modes, CONTINUOUS, CHANGE or TRIGGER.  Each 
command can be individually enabled or disabled.  The start up state is given by the
\texttt{.enable} member of the command table in the configuration file.  After that the command
can be enabled by setting the corresponding bit in the tag that is created.  This tag
will be named [port name]\_cmd\_en, and is an array of BOOLs that each correspond to the
commands in the order in which they were added in the configuration file.

If the command is set to CONTINUOUS then it will be sent at some multiple of the ports
scantime.  This multiple is given by the .interval member of the command table.  For
example, if the scantime on the port is set to 500 mS and the interval for a command is set to 3, that command will be sent every 1.5 seconds.

If the command is set to CHANGE then it will be sent when the data tag that is associated with it changes.  This is only applicable for function codes that write data.
TRIGGER is the mode that uses a tag to trigger the sending of the command.  This allows
other logic in the system to decide when to send a command.

